%!TEX root=cs441S2016_lab3.tex

\input{labspre.tex}

\usepackage[compact]{titlesec}

\begin{document}
\MYTITLE{Laboratory Assignment Three: Exploring and Evaluating Parallel Computation}
\MYHEADERS{Laboratory Assignment Three}{Due: February 15, 2016}

\section*{Introduction}

Since processes, or ``programs in execution'', form the building blocks of distributed systems, it is important to learn
how to create and run processes on an operating system. To ensure that distributed systems exhibit the best throughput
and response times, it is also critical that these processes are, whenever it is appropriate to do so, executed in a
parallel fashion. In this assignment, you will use two techniques for parallelizing computations on a single node. Next,
you will conduct an experiment to determine how response time changes as the parameters of the parallel computation are
modified.  Finally, using either Markdown or the \LaTeX~text formatting language, you will write a report explaining the
performance results that you identified during this assignment.

\section*{Review Your Textbook}

Before starting this assignment, you should again read the content in Chapters 1 and 2 of your textbook so that you have
a strong understanding of the basic concepts in the field of distributed systems. Next, please read the content in
Section 3.1, paying particularly close attention to the content about processes and threads. As you review Section 3.1,
make sure that you understand the trade-offs associated with using parallel threads and processes in a software system.
Students who want to learn more about the {\tt xargs} and {\tt parallel} programs that are part of this laboratory
assignment are also encouraged to use the {\tt man} program to read their manual pages.

\section*{}

Since you configured your connection to the ``share'' repository for this course in a previous laboratory assignment,
you should be able to change into the {\tt cs441S2016-share} directory and type the ``{\tt git pull}'' command to gain
access to the many files that are needed to complete this assignment.






\section*{Evaluating the Parallel Computation}


You should organize all of your empirical results into tables of data. While not absolutely required, you may consider
preparing graphs of your results using the R language for statistical computation. Next, you should analyze the results
in attempt to find and explain patterns in the data. Overall, what do your results show you about the cost of performing
parallel image manipulation? Once you are finished running these experiments, can you identify any ways in which you
could have improved the performance of the parallel computations? Finally, you should write a detailed report, using
either the Markdown or the \LaTeX~text formatting language, that introduces the design of your experiment and your
research questions, explains how you conducted the experiments and analyzed the results, and then presents and analyzes
the results.  Please see the course instructor if you have questions about how to write a scientific report using
\LaTeX~or Markdown.

\section*{Summary of the Required Deliverables}

This assignment invites you to submit printed and signed versions of the following deliverables. Additionally,
all of these deliverables must be in a {\tt cs441S2016-<your user name>} repository that you created for this class;
please make sure that you share this repository with the instructor.

\begin{enumerate}

    \item The commented source code of the Java classes that perform client-server-based file transfer.

    \item Using both text and diagrams, a description of client-server-based file transfer with sockets.

    \item A description of the challenges associated with performing client-server communication.

    \item A comparison of file transfer methods that use ``centralized'' and ``decentralized'' approaches.

    \item A detailed paper that reports on the empirical results arising from performing file transfers.

    \item A description of the challenges that you encountered when completing this assignment.

\end{enumerate}

In adherence to the Honor Code, students should complete this assignment on an individual basis. While it is appropriate
for students in this class to have high-level conversations about the assignment, it is necessary to distinguish
carefully between the student who discusses the principles underlying a problem with others and the student who produces
assignments that are identical to, or merely variations on, someone else's work.  With the exception of the provided
source code, deliverables that are otherwise nearly identical to the work of others will be taken as evidence of
violating the \mbox{Honor Code}. This means that, for instance, all of the other comments, source code, data, and
written reports should be the original work of the student completing this assignment. Students who have questions about
the Honor Code and how it applies to this laboratory assignment should schedule a meeting with the course instructor
before this assignment's due date.

\end{document}
