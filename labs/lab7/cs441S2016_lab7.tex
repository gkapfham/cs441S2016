\input{labspre.tex}

\usepackage[compact]{titlesec}

\begin{document}
\MYTITLE{Laboratory Assignment Seven: Understanding Remote Method Invocation in Java}
\MYHEADERS{Laboratory Assignment Seven}{Due: March 14, 2016}

\section*{Introduction}

% There are certain situations in which it may not be best to create a client-server-based distributed system that uses
% unicasting for the exchange of messages. For instance, the need for an alternative to unicasting can arise when a single
% node in the distributed system must quickly communicate a message to many nodes. Unicasting may also be an
% unsuitable solution when a client is attempting to locate a server that migrates around a local-area network. This
% laboratory assignment will explore multicasting as an alternative to unicasting, giving students the opportunity to
% investigate the trade-offs of this alternative to communicating in a ``point-to-point'' fashion.  In addition to running
% a multicast system supporting communication between several nodes, you will write a report, using either Markdown
% or \LaTeX, examining the trade-offs and applications of multicasting.

In the Java programming language, it is common to implement a distributed object-based system using the remote method
invocation (RMI) classes. In this laboratory assignment, you will download and use a Java system that is derived from
the example provided at \url{https://github.com/pothoven/simple-rmi}. Once you have correctly compiled and configured
this system, you will run it and explore all of the features that it provides. Next, you will conduct an experiment to
measure the response time associated with the file upload feature that the {\tt simple-rmi} system furnishes. Finally,
you will add one new feature to the provided system. In addition to running these experiments and adding a new feature,
you will write a report, using either Markdown or \LaTeX, that presents on your performance results and reflects on your
implementation experiences.

\section*{Review Your Textbook}

Before starting this assignment, you should review Chapters 1 through 4 of your textbook. If you have a question about
this content, then please resolve it before starting this laboratory assignment. Next, you should examine the material
in Sections 10.1 and 10.2 that discusses the features commonly found in a distributed object-based system. Additionally,
students should pay close attention to the paragraphs in Section 10.3 that focus on remote method invocation in the Java
programming language and the Java distributed-object model. Although it is not the emphasis of this assignment, students
who want to learn about security protocols and how they are used in distributed object-based systems in Java should
study the content in Section 10.8 and the security policy file in the provided implementation.  Please see the
instructor if you have any questions.

\section*{Exploring Remote Method Invocation in Java}

Since you configured your connection to the ``share'' repository for this course in a previous laboratory assignment,
you should be able to change into the {\tt cs441S2016-share/} directory and type the ``{\tt git pull}'' command to gain
access to the many Java classes needed to complete this assignment. Please recall that the source code that you receive
for this laboratory assignment is derived from the Java system that is available at
\url{https://github.com/pothoven/simple-rmi}; students who would like to learn more about this system are encouraged to
visit the referenced GitHub page.

Once you have carefully studied the two provided Java classes, you should attempt to compile and run them in your
terminal window. Please note that this system will only work correctly if you first run the {\tt
MulticastSocketReceiver} and you then execute the {\tt MulticastSocketSender}. After running these two programs, you
should record the output that they produce. Next, please try to run one, two, three, four, and five separate versions of
the {\tt MulticastSocketReceiver} on different computers in the Alden Hall network. You can accomplish this task by
using {\tt ssh} to connect to a remote computer in a separate terminal window and then running the {\tt
  MulticastSocketReceiver} from this location. Next, please find the location in the source code that declares the {\tt
String INET\_ADDR} variable. What is the purpose of this variable? Is it possible for this program to work if you change
this address to a different value? What are some values that seem to work correctly?


\section*{Summary of the Required Deliverables}

This assignment invites you to submit printed and signed versions of the following deliverables. Additionally,
all of these deliverables must be in a {\tt cs441S2016-<your user name>} repository that you created for this class;
please make sure that you share this repository with the instructor.

\vspace*{-.1in}

\begin{enumerate}
  \itemsep 0em

  \item The well-commented source code of the Java class called {\tt MulticastSocketSender}.

  \item The well-commented source code of the Java class called {\tt MulticastSocketReceiver}.

  \item Using both text and diagrams, a description of client-server communication with multicasting.

  \item A report that explicitly compares and contrasts multicast and unicast remote communication.

  \item A document that summarizes the different types of multicasting systems used in production.

  \item A paper that responds to the other questions that this assignment poses about multicasting.

  \item A reflection on the challenges that you encountered when completing this assignment.

\end{enumerate}

\vspace*{-.1in}

% With the exception of the provided source code, deliverables that are otherwise nearly identical to the work of others
% will be taken as evidence of violating the \mbox{Honor Code}.

In adherence to the Honor Code, students should complete this assignment on an individual basis. While it is appropriate
for students in this class to have high-level conversations about the assignment, it is necessary to distinguish
carefully between the student who discusses the principles underlying a problem with others and the student who produces
assignments that are identical to, or merely variations on, someone else's work.  With the exception of the provided
source code, deliverables that are otherwise nearly identical to the work of others will be taken as evidence of
violating the \mbox{Honor Code}. This means that, for instance, all of the other comments, source code, data, and
written reports should be the original work of the student completing this assignment.

% Students who have questions about the Honor Code and how it applies to this laboratory assignment should schedule a
% meeting with the course instructor before this assignment's due date.

\end{document}
