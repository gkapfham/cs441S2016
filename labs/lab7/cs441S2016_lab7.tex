%!TEX root=cs440S2014-lab7.tex
% mainfile: cs440S2014-lab7.tex 
% CS 580 style
% Typical usage (all UPPERCASE items are optional):
%       \input 580pre
%       \begin{document}
%       \MYTITLE{Title of document, e.g., Lab 1\\Due ...}
%       \MYHEADERS{short title}{other running head, e.g., due date}
%       \PURPOSE{Description of purpose}
%       \SUMMARY{Very short overview of assignment}
%       \DETAILS{Detailed description}
%         \SUBHEAD{if needed} ...
%         \SUBHEAD{if needed} ...
%          ...
%       \HANDIN{What to hand in and how}
%       \begin{checklist}
%       \item ...
%       \end{checklist}
% There is no need to include a "\documentstyle."
% However, there should be an "\end{document}."
%
%===========================================================
\documentclass[11pt,twoside,titlepage]{article}
%%NEED TO ADD epsf!!
\usepackage{threeparttop}
\usepackage{graphicx}
\usepackage{latexsym}
\usepackage{color}
\usepackage{listings}
\usepackage{fancyvrb}
%\usepackage{pgf,pgfarrows,pgfnodes,pgfautomata,pgfheaps,pgfshade}
\usepackage{tikz}
\usepackage[normalem]{ulem}
\tikzset{
    %Define standard arrow tip
%    >=stealth',
    %Define style for boxes
    oval/.style={
           rectangle,
           rounded corners,
           draw=black, very thick,
           text width=6.5em,
           minimum height=2em,
           text centered},
    % Define arrow style
    arr/.style={
           ->,
           thick,
           shorten <=2pt,
           shorten >=2pt,}
}
\usepackage[noend]{algorithmic}
\usepackage[noend]{algorithm}
\newcommand{\bfor}{{\bf for\ }}
\newcommand{\bthen}{{\bf then\ }}
\newcommand{\bwhile}{{\bf while\ }}
\newcommand{\btrue}{{\bf true\ }}
\newcommand{\bfalse}{{\bf false\ }}
\newcommand{\bto}{{\bf to\ }}
\newcommand{\bdo}{{\bf do\ }}
\newcommand{\bif}{{\bf if\ }}
\newcommand{\belse}{{\bf else\ }}
\newcommand{\band}{{\bf and\ }}
\newcommand{\breturn}{{\bf return\ }}
\newcommand{\mod}{{\rm mod}}
\renewcommand{\algorithmiccomment}[1]{$\rhd$ #1}
\newenvironment{checklist}{\par\noindent\hspace{-.25in}{\bf Checklist:}\renewcommand{\labelitemi}{$\Box$}%
\begin{itemize}}{\end{itemize}}
\pagestyle{threepartheadings}
\usepackage{url}
\usepackage{wrapfig}
% removing the standard hyperref to avoid the horrible boxes
%\usepackage{hyperref}
\usepackage[hidelinks]{hyperref}
% added in the dtklogos for the bibtex formatting
\usepackage{dtklogos}
%=========================
% One-inch margins everywhere
%=========================
\setlength{\topmargin}{0in}
\setlength{\textheight}{8.5in}
\setlength{\oddsidemargin}{0in}
\setlength{\evensidemargin}{0in}
\setlength{\textwidth}{6.5in}
%===============================
%===============================
% Macro for document title:
%===============================
\newcommand{\MYTITLE}[1]%
   {\begin{center}
     \begin{center}
     \bf
     CMPSC 440\\Operating Systems\\
     Spring 2014
     \medskip
     \end{center}
     \bf
     #1
     \end{center}
}
%================================
% Macro for headings:
%================================
\newcommand{\MYHEADERS}[2]%
   {\lhead{#1}
    \rhead{#2}
    %\immediate\write16{}
    %\immediate\write16{DATE OF HANDOUT?}
    %\read16 to \dateofhandout
    \def \dateofhandout {April 7, 2014}
    \lfoot{\sc Handed out on \dateofhandout}
    %\immediate\write16{}
    %\immediate\write16{HANDOUT NUMBER?}
    %\read16 to\handoutnum
    \def \handoutnum {9}
    \rfoot{Handout \handoutnum}
   }

%================================
% Macro for bold italic:
%================================
\newcommand{\bit}[1]{{\textit{\textbf{#1}}}}

%=========================
% Non-zero paragraph skips.
%=========================
\setlength{\parskip}{1ex}

%=========================
% Create various environments:
%=========================
\newcommand{\PURPOSE}{\par\noindent\hspace{-.25in}{\bf Purpose:\ }}
\newcommand{\SUMMARY}{\par\noindent\hspace{-.25in}{\bf Summary:\ }}
\newcommand{\DETAILS}{\par\noindent\hspace{-.25in}{\bf Details:\ }}
\newcommand{\HANDIN}{\par\noindent\hspace{-.25in}{\bf Hand in:\ }}
\newcommand{\SUBHEAD}[1]{\bigskip\par\noindent\hspace{-.1in}{\sc #1}\\}
%\newenvironment{CHECKLIST}{\begin{itemize}}{\end{itemize}}


\usepackage[compact]{titlesec}

\begin{document}
\MYTITLE{Laboratory Assignment Seven: Understanding Remote Method Invocation in Java}
\MYHEADERS{Laboratory Assignment Seven}{Due: March 14, 2016}

\section*{Introduction}

% There are certain situations in which it may not be best to create a client-server-based distributed system that uses
% unicasting for the exchange of messages. For instance, the need for an alternative to unicasting can arise when a single
% node in the distributed system must quickly communicate a message to many nodes. Unicasting may also be an
% unsuitable solution when a client is attempting to locate a server that migrates around a local-area network. This
% laboratory assignment will explore multicasting as an alternative to unicasting, giving students the opportunity to
% investigate the trade-offs of this alternative to communicating in a ``point-to-point'' fashion.  In addition to running
% a multicast system supporting communication between several nodes, you will write a report, using either Markdown
% or \LaTeX, examining the trade-offs and applications of multicasting.

In the Java programming language, it is common to implement a distributed object-based system using the remote method
invocation (RMI) classes. In this laboratory assignment, you will download and use a Java system that is derived from
the example provided at \url{https://github.com/pothoven/simple-rmi}. Once you have correctly compiled and configured
this system, you will run it and explore all of the features that it provides. Next, you will conduct an experiment to
measure the response time associated with the file upload feature that the {\tt simple-rmi} system furnishes. Finally,
you will add one new feature to the provided system. In addition to running these experiments and adding a new feature,
you will write a report, using either Markdown or \LaTeX, that presents on your performance results and reflects on your
implementation experiences.

\section*{Review Your Textbook}

Before starting this assignment, you should review Chapters 1 through 4 of your textbook. If you have a question about
this content, then please resolve it before starting this laboratory assignment. Next, you should examine the material
in Sections 10.1 and 10.2 that discusses the features commonly found in a distributed object-based system. Additionally,
students should pay close attention to the paragraphs in Section 10.3 that focus on remote method invocation in the Java
programming language and the Java distributed-object model. Although it is not the emphasis of this assignment, students
who want to learn about security protocols and how they are used in distributed object-based systems in Java should
study the content in Section 10.8 and the security policy file in the provided implementation.  Please see the
instructor if you have any questions.

\section*{Exploring Remote Method Invocation in Java}

Since you configured your connection to the ``share'' repository for this course in a previous laboratory assignment,
you should be able to change into the {\tt cs441S2016-share/} directory and type the ``{\tt git pull}'' command to gain
access to the many Java classes needed to complete this assignment. Please recall that the source code that you receive
for this laboratory assignment is derived from the Java system that is available at
\url{https://github.com/pothoven/simple-rmi}; students who would like to learn more about this system are encouraged to
visit the referenced GitHub page.

\section*{Applications and Enhancements of Java RMI}

Java RMI serves as the ``backbone'' for a wide variety of different types of distributed systems. Additionally, there
are many ways in which researchers and practitioners have extended Java RMI so that it provides new features that are
not a part of the standard distribution. As the final portion of this laboratory assignment, you should search the ACM
Digital Library ---- available from \url{http://dl.acm.org/} --- and download and read two papers. After carefully
studying the papers, you should write a short one or two paragraph review of each paper that highlights its key
contributions. Finally, your review of the paper should comment on how the implementation that you used in this
assignment is similar to and different from the one presented in the paper.

\section*{Summary of the Required Deliverables}

This assignment invites you to submit printed and signed versions of the following deliverables. Additionally,
all of these deliverables must be in a {\tt cs441S2016-<your user name>} repository that you created for this class;
please make sure that you share this repository with the instructor.

\vspace*{-.1in}

\begin{enumerate}
  \itemsep 0em

  \item The well-commented source code of all of the provided Java classes.

  \item Using both text and diagrams, a description of client-server communication with Java RMI.

  \item A report that explicitly compares and contrasts multicast and unicast remote communication.

  \item A document that summarizes the different types of multicasting systems used in production.

  \item A paper that responds to the other questions that this assignment poses about multicasting.

  \item A reflection on the challenges that you encountered when completing this assignment.

\end{enumerate}

\vspace*{-.1in}

% With the exception of the provided source code, deliverables that are otherwise nearly identical to the work of others
% will be taken as evidence of violating the \mbox{Honor Code}.

In adherence to the Honor Code, students should complete this assignment on an individual basis. While it is appropriate
for students in this class to have high-level conversations about the assignment, it is necessary to distinguish
carefully between the student who discusses the principles underlying a problem with others and the student who produces
assignments that are identical to, or merely variations on, someone else's work.  With the exception of the provided
source code, deliverables that are otherwise nearly identical to the work of others will be taken as evidence of
violating the \mbox{Honor Code}. This means that, for instance, all of the other comments, source code, data, and
written reports should be the original work of the student completing this assignment.

% Students who have questions about the Honor Code and how it applies to this laboratory assignment should schedule a
% meeting with the course instructor before this assignment's due date.

\end{document}
