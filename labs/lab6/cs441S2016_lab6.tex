%!TEX root=cs440S2014-lab7.tex
% mainfile: cs440S2014-lab7.tex 
% CS 580 style
% Typical usage (all UPPERCASE items are optional):
%       \input 580pre
%       \begin{document}
%       \MYTITLE{Title of document, e.g., Lab 1\\Due ...}
%       \MYHEADERS{short title}{other running head, e.g., due date}
%       \PURPOSE{Description of purpose}
%       \SUMMARY{Very short overview of assignment}
%       \DETAILS{Detailed description}
%         \SUBHEAD{if needed} ...
%         \SUBHEAD{if needed} ...
%          ...
%       \HANDIN{What to hand in and how}
%       \begin{checklist}
%       \item ...
%       \end{checklist}
% There is no need to include a "\documentstyle."
% However, there should be an "\end{document}."
%
%===========================================================
\documentclass[11pt,twoside,titlepage]{article}
%%NEED TO ADD epsf!!
\usepackage{threeparttop}
\usepackage{graphicx}
\usepackage{latexsym}
\usepackage{color}
\usepackage{listings}
\usepackage{fancyvrb}
%\usepackage{pgf,pgfarrows,pgfnodes,pgfautomata,pgfheaps,pgfshade}
\usepackage{tikz}
\usepackage[normalem]{ulem}
\tikzset{
    %Define standard arrow tip
%    >=stealth',
    %Define style for boxes
    oval/.style={
           rectangle,
           rounded corners,
           draw=black, very thick,
           text width=6.5em,
           minimum height=2em,
           text centered},
    % Define arrow style
    arr/.style={
           ->,
           thick,
           shorten <=2pt,
           shorten >=2pt,}
}
\usepackage[noend]{algorithmic}
\usepackage[noend]{algorithm}
\newcommand{\bfor}{{\bf for\ }}
\newcommand{\bthen}{{\bf then\ }}
\newcommand{\bwhile}{{\bf while\ }}
\newcommand{\btrue}{{\bf true\ }}
\newcommand{\bfalse}{{\bf false\ }}
\newcommand{\bto}{{\bf to\ }}
\newcommand{\bdo}{{\bf do\ }}
\newcommand{\bif}{{\bf if\ }}
\newcommand{\belse}{{\bf else\ }}
\newcommand{\band}{{\bf and\ }}
\newcommand{\breturn}{{\bf return\ }}
\newcommand{\mod}{{\rm mod}}
\renewcommand{\algorithmiccomment}[1]{$\rhd$ #1}
\newenvironment{checklist}{\par\noindent\hspace{-.25in}{\bf Checklist:}\renewcommand{\labelitemi}{$\Box$}%
\begin{itemize}}{\end{itemize}}
\pagestyle{threepartheadings}
\usepackage{url}
\usepackage{wrapfig}
% removing the standard hyperref to avoid the horrible boxes
%\usepackage{hyperref}
\usepackage[hidelinks]{hyperref}
% added in the dtklogos for the bibtex formatting
\usepackage{dtklogos}
%=========================
% One-inch margins everywhere
%=========================
\setlength{\topmargin}{0in}
\setlength{\textheight}{8.5in}
\setlength{\oddsidemargin}{0in}
\setlength{\evensidemargin}{0in}
\setlength{\textwidth}{6.5in}
%===============================
%===============================
% Macro for document title:
%===============================
\newcommand{\MYTITLE}[1]%
   {\begin{center}
     \begin{center}
     \bf
     CMPSC 440\\Operating Systems\\
     Spring 2014
     \medskip
     \end{center}
     \bf
     #1
     \end{center}
}
%================================
% Macro for headings:
%================================
\newcommand{\MYHEADERS}[2]%
   {\lhead{#1}
    \rhead{#2}
    %\immediate\write16{}
    %\immediate\write16{DATE OF HANDOUT?}
    %\read16 to \dateofhandout
    \def \dateofhandout {April 7, 2014}
    \lfoot{\sc Handed out on \dateofhandout}
    %\immediate\write16{}
    %\immediate\write16{HANDOUT NUMBER?}
    %\read16 to\handoutnum
    \def \handoutnum {9}
    \rfoot{Handout \handoutnum}
   }

%================================
% Macro for bold italic:
%================================
\newcommand{\bit}[1]{{\textit{\textbf{#1}}}}

%=========================
% Non-zero paragraph skips.
%=========================
\setlength{\parskip}{1ex}

%=========================
% Create various environments:
%=========================
\newcommand{\PURPOSE}{\par\noindent\hspace{-.25in}{\bf Purpose:\ }}
\newcommand{\SUMMARY}{\par\noindent\hspace{-.25in}{\bf Summary:\ }}
\newcommand{\DETAILS}{\par\noindent\hspace{-.25in}{\bf Details:\ }}
\newcommand{\HANDIN}{\par\noindent\hspace{-.25in}{\bf Hand in:\ }}
\newcommand{\SUBHEAD}[1]{\bigskip\par\noindent\hspace{-.1in}{\sc #1}\\}
%\newenvironment{CHECKLIST}{\begin{itemize}}{\end{itemize}}


\usepackage[compact]{titlesec}

\begin{document}
\MYTITLE{Laboratory Assignment Six: Exploring the Trade-offs of Multicast Communication}
\MYHEADERS{Laboratory Assignment Six}{Due: March 7, 2016}

\section*{Introduction}

There are certain situations in which it may not be best to create a client-server-based distributed system that uses
unicasting for the exchange of messages. For instance, the need for an alternative to unicasting can arise when a single
node in the distributed system must quickly communicate a message to many nodes. Unicasting may also be an
unsuitable solution when a client is attempting to locate a server that migrates around a local-area network. This
laboratory assignment will explore multicasting as an alternative to unicasting, giving students the opportunity to
investigate the trade-offs of this alternative to communicating in a ``point-to-point'' fashion.  In addition to running
a multicast system supporting communication between several nodes, you will write a report, using either Markdown
or \LaTeX, examining the trade-offs and applications of multicasting.

\section*{Review Your Textbook}

Before starting this laboratory assignment, you should review the content in Chapters 1 through 3 of your textbook. If
you have a question about this content, then please resolve it before starting this laboratory assignment. Next, you
should examine the content in Section 4.1 that discusses the protocols commonly used in distributed systems, paying
particularly close attention to the material about connection-oriented and connectionless protocols. It is also
important for you to study the content in Section 4.5 that talks about the various approaches to multicast
communication. Students who are interested in studying a more advanced treatment of ways to achieve reliable
multicasting are encouraged to read Section 8.4. Please see the instructor if you have any questions.

\section*{Exploring Multicast Communication}

Since you configured your connection to the ``share'' repository for this course in a previous laboratory assignment,
you should be able to change into the {\tt cs441S2016-share/} directory and type the ``{\tt git pull}'' command to gain
access to the two Java classes needed to complete this assignment.

\begin{sloppypar}
  Now, please find the {\tt labs/lab6/} directory in our course's repository so that you can study and understand the
  source code for the files called {\tt MulticastSocketReceiver.java} and {\tt MulticastSocketSender.java}. What are the
  key aspects of how these two Java classes support multicast communication? For instance, you will notice that these
  Java classes specify a ``port'' --- and, yet, do not give a ``host'' as you saw in the previous assignments. To learn more
  about why this is the case and to better understand the trade-offs associated with this approach, you should read the
  Java documentation for the {\tt java.net.DatagramPacket}, {\tt java.net.InetAddress}, and {\tt
  java.net.MulticastSocket} classes. In particular, you should investigate what it means for the {\tt DatagramPacket}
  class to, according to its documentation, ``implement a connectionless packet delivery service''.  Finally, can you
  draw a technical diagram that illustrates the general interaction between a client and a server, customized for these
  two classes that perform multicast communication with sockets? Please see the course instructor if you have
  questions about these issues.
\end{sloppypar}

Once you have carefully studied the two provided Java classes, you should attempt to compile and run them in your
terminal window. Please note that this system will only work correctly if you first run the {\tt
MulticastSocketReceiver} and you then execute the {\tt MulticastSocketSender}. After running these two programs, you
should record the output that they produce. Next, please try to run one, two, three, four, and five separate versions of
the {\tt MulticastSocketReceiver} on different computers in the Alden Hall network. You can accomplish this task by
using {\tt ssh} to connect to a remote computer in a separate terminal window and then running the {\tt
  MulticastSocketReceiver} from this location. Next, please find the location in the source code that declares the {\tt
String INET\_ADDR} variable. What is the purpose of this variable? Is it possible for this program to work if you change
this address to a different value? What are some values that seem to work correctly?

There are many real-world applications of multicasting in a distributed system. For instance, the transmission of stock
quotes to many listening computers is a natural application for multicasting. Using the ACM Digital Library, please
locate and download the article ``An Internet multicast system for the stock market'' by N.\ F.\ Maxemchuk and D.\ H.\
Shur. How is the approach to multicasting that this paper presents different than the one that you have used in this
laboratory assignment? Would it be a good idea to use the code from this laboratory assignment in a multicasting system
for the stock market? Why or why not? In addition to reading and reporting on this paper, you should also download and
read one more paper from the ACM Digital Library. Students should focus on finding an additional article that presents
either a real-world development and application of multicasting or an empirical characterization of how multicasting is
used in a real-world setting. Please see the instructor if you have questions about this part of the assignment.

\section*{Summary of the Required Deliverables}

This assignment invites you to submit printed and signed versions of the following deliverables. Additionally,
all of these deliverables must be in a {\tt cs441S2016-<your user name>} repository that you created for this class;
please make sure that you share this repository with the instructor.

\vspace*{-.1in}

\begin{enumerate}
  \itemsep 0em

  \item The well-commented source code of the Java class called {\tt MulticastSocketSender}.

  \item The well-commented source code of the Java class called {\tt MulticastSocketReceiver}.

  \item Using both text and diagrams, a description of client-server communication with multicasting.

  \item A report that explicitly compares and contrasts multicast and unicast remote communication.

  \item A document that summarizes the different types of multicasting systems used in production.

  \item A paper that responds to the other questions that this assignment poses about multicasting.

  \item A reflection on the challenges that you encountered when completing this assignment.

\end{enumerate}

\vspace*{-.1in}

% With the exception of the provided source code, deliverables that are otherwise nearly identical to the work of others
% will be taken as evidence of violating the \mbox{Honor Code}.

In adherence to the Honor Code, students should complete this assignment on an individual basis. While it is appropriate
for students in this class to have high-level conversations about the assignment, it is necessary to distinguish
carefully between the student who discusses the principles underlying a problem with others and the student who produces
assignments that are identical to, or merely variations on, someone else's work.  With the exception of the provided
source code, deliverables that are otherwise nearly identical to the work of others will be taken as evidence of
violating the \mbox{Honor Code}. This means that, for instance, all of the other comments, source code, data, and
written reports should be the original work of the student completing this assignment.

% Students who have questions about the Honor Code and how it applies to this laboratory assignment should schedule a
% meeting with the course instructor before this assignment's due date.

\end{document}
