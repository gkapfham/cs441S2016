%!TEX root=cs440S2014-lab7.tex
% mainfile: cs440S2014-lab7.tex 
% CS 580 style
% Typical usage (all UPPERCASE items are optional):
%       \input 580pre
%       \begin{document}
%       \MYTITLE{Title of document, e.g., Lab 1\\Due ...}
%       \MYHEADERS{short title}{other running head, e.g., due date}
%       \PURPOSE{Description of purpose}
%       \SUMMARY{Very short overview of assignment}
%       \DETAILS{Detailed description}
%         \SUBHEAD{if needed} ...
%         \SUBHEAD{if needed} ...
%          ...
%       \HANDIN{What to hand in and how}
%       \begin{checklist}
%       \item ...
%       \end{checklist}
% There is no need to include a "\documentstyle."
% However, there should be an "\end{document}."
%
%===========================================================
\documentclass[11pt,twoside,titlepage]{article}
%%NEED TO ADD epsf!!
\usepackage{threeparttop}
\usepackage{graphicx}
\usepackage{latexsym}
\usepackage{color}
\usepackage{listings}
\usepackage{fancyvrb}
%\usepackage{pgf,pgfarrows,pgfnodes,pgfautomata,pgfheaps,pgfshade}
\usepackage{tikz}
\usepackage[normalem]{ulem}
\tikzset{
    %Define standard arrow tip
%    >=stealth',
    %Define style for boxes
    oval/.style={
           rectangle,
           rounded corners,
           draw=black, very thick,
           text width=6.5em,
           minimum height=2em,
           text centered},
    % Define arrow style
    arr/.style={
           ->,
           thick,
           shorten <=2pt,
           shorten >=2pt,}
}
\usepackage[noend]{algorithmic}
\usepackage[noend]{algorithm}
\newcommand{\bfor}{{\bf for\ }}
\newcommand{\bthen}{{\bf then\ }}
\newcommand{\bwhile}{{\bf while\ }}
\newcommand{\btrue}{{\bf true\ }}
\newcommand{\bfalse}{{\bf false\ }}
\newcommand{\bto}{{\bf to\ }}
\newcommand{\bdo}{{\bf do\ }}
\newcommand{\bif}{{\bf if\ }}
\newcommand{\belse}{{\bf else\ }}
\newcommand{\band}{{\bf and\ }}
\newcommand{\breturn}{{\bf return\ }}
\newcommand{\mod}{{\rm mod}}
\renewcommand{\algorithmiccomment}[1]{$\rhd$ #1}
\newenvironment{checklist}{\par\noindent\hspace{-.25in}{\bf Checklist:}\renewcommand{\labelitemi}{$\Box$}%
\begin{itemize}}{\end{itemize}}
\pagestyle{threepartheadings}
\usepackage{url}
\usepackage{wrapfig}
% removing the standard hyperref to avoid the horrible boxes
%\usepackage{hyperref}
\usepackage[hidelinks]{hyperref}
% added in the dtklogos for the bibtex formatting
\usepackage{dtklogos}
%=========================
% One-inch margins everywhere
%=========================
\setlength{\topmargin}{0in}
\setlength{\textheight}{8.5in}
\setlength{\oddsidemargin}{0in}
\setlength{\evensidemargin}{0in}
\setlength{\textwidth}{6.5in}
%===============================
%===============================
% Macro for document title:
%===============================
\newcommand{\MYTITLE}[1]%
   {\begin{center}
     \begin{center}
     \bf
     CMPSC 440\\Operating Systems\\
     Spring 2014
     \medskip
     \end{center}
     \bf
     #1
     \end{center}
}
%================================
% Macro for headings:
%================================
\newcommand{\MYHEADERS}[2]%
   {\lhead{#1}
    \rhead{#2}
    %\immediate\write16{}
    %\immediate\write16{DATE OF HANDOUT?}
    %\read16 to \dateofhandout
    \def \dateofhandout {April 7, 2014}
    \lfoot{\sc Handed out on \dateofhandout}
    %\immediate\write16{}
    %\immediate\write16{HANDOUT NUMBER?}
    %\read16 to\handoutnum
    \def \handoutnum {9}
    \rfoot{Handout \handoutnum}
   }

%================================
% Macro for bold italic:
%================================
\newcommand{\bit}[1]{{\textit{\textbf{#1}}}}

%=========================
% Non-zero paragraph skips.
%=========================
\setlength{\parskip}{1ex}

%=========================
% Create various environments:
%=========================
\newcommand{\PURPOSE}{\par\noindent\hspace{-.25in}{\bf Purpose:\ }}
\newcommand{\SUMMARY}{\par\noindent\hspace{-.25in}{\bf Summary:\ }}
\newcommand{\DETAILS}{\par\noindent\hspace{-.25in}{\bf Details:\ }}
\newcommand{\HANDIN}{\par\noindent\hspace{-.25in}{\bf Hand in:\ }}
\newcommand{\SUBHEAD}[1]{\bigskip\par\noindent\hspace{-.1in}{\sc #1}\\}
%\newenvironment{CHECKLIST}{\begin{itemize}}{\end{itemize}}


\usepackage[compact]{titlesec}

\begin{document}
\MYTITLE{Laboratory Assignment Six: Exploring the Trade-offs of Multicast Communication}
\MYHEADERS{Laboratory Assignment Six}{Due: March 7, 2016}

\section*{Introduction}

There are certain situations in which it is not feasible to use a client-server based distributed system that uses
unicasting for the exchange of messages. For instance, the need for an alternative to unicasting can arise when a single
node in the distributed system needs to quickly communicate a message to many nodes. Unicasting may also be an
unsuitable solution when a client is attempting to locate a server that migrates around a local-area network. This
laboratory assignment will explore multicasting as an alternative to unicasting, giving students the opportunity to
investigate the trade-offs of this alternative to ``point-to-point'' communication.  In addition to running a multicast
system supporting communication between several nodes, you will write a short report, using either Markdown or the
\LaTeX, that discusses the trade-offs associated with multicasting.

\section*{Review Your Textbook}

Before starting this laboratory assignment, you and your partner should review and discuss the content in Chapters 1
through 3 of your textbook. If one member of your partnership has questions about this content, then please resolve them
before starting this laboratory assignment. Next, you should collaboratively examine the material in Section 4.1, paying
particularly close attention to the technical digram in Figure 4-6 and the ten steps associated with performing a remote
procedure call, as given on page 129 of your textbook. Please see the instructor if you have any questions.

\section*{Access the Course Repository}

Since you configured your connection to the ``share'' repository for this course in a previous laboratory assignment,
you should be able to change into the {\tt cs441S2016-share} directory and type the ``{\tt git pull}'' command to gain
access to the two Java classes needed to complete this assignment.

Now, please find the {\tt labs/lab2/} directory in our course's repository so that you can study and understand the
source code for the files called {\tt FileSocketClient} and {\tt FileSocketServer}. What are the key aspects of how
these two Java classes support file transfer? For instance, you will notice that these Java classes specify items like a
``host'' and a ``port'' --- as in the previous laboratory assignment, make sure that you understand the meanings of
these terms. Can you draw a technical diagram that illustrates the general interaction between a client and a server,
customized for these two classes that uses sockets? Please see the instructor if you have questions about these issues.

\section*{Exploring Multicast Communication}



\section*{Summary of the Required Deliverables}

This assignment invites you to submit printed and signed versions of the following deliverables. Additionally,
all of these deliverables must be in the repository that you created for this assignment.

\vspace*{-.15in}

\begin{enumerate}
  \itemsep 0em

    \item The well-commented source code of the Java classes that form the two ``useful'' benchmarks.

    \item The well-commented source code of the Java classes for the four ``baseline'' benchmarks.

    \item Using both text and diagrams, a description of client-server communication with XML-RPC.

    \item A detailed paper that reports on the empirical results arising from the use of the benchmarks.

    \item A description of the challenges that you encountered when completing this assignment.

    \item A detailed listing of the tasks completed by each member of your partnership.

\end{enumerate}

\vspace*{-.15in}

% With the exception of the provided source code, deliverables that are otherwise nearly identical to the work of others
% will be taken as evidence of violating the \mbox{Honor Code}.

In adherence to the Honor Code, students should complete this assignment on an group basis. While it is appropriate for
students in this class to have high-level conversations about the assignment, it is necessary to distinguish carefully
between the student who discusses with others the principles underlying a problem and the student who produces
assignments that are identical to, or merely variations on, someone else's work.  This means that, for instance, all of
the other comments, source code, data, and written reports --- with the exception of the code that the instructor
provided through the ``share'' repository --- should be the original work of the members of the partnership.  Students
who have questions about the Honor Code and how it applies to this assignment should schedule a meeting with the course
instructor before this assignment's due date.

\end{document}
