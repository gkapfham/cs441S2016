%!TEX root=cs441S2016_lab4.tex

%!TEX root=cs440S2014-lab7.tex
% mainfile: cs440S2014-lab7.tex 
% CS 580 style
% Typical usage (all UPPERCASE items are optional):
%       \input 580pre
%       \begin{document}
%       \MYTITLE{Title of document, e.g., Lab 1\\Due ...}
%       \MYHEADERS{short title}{other running head, e.g., due date}
%       \PURPOSE{Description of purpose}
%       \SUMMARY{Very short overview of assignment}
%       \DETAILS{Detailed description}
%         \SUBHEAD{if needed} ...
%         \SUBHEAD{if needed} ...
%          ...
%       \HANDIN{What to hand in and how}
%       \begin{checklist}
%       \item ...
%       \end{checklist}
% There is no need to include a "\documentstyle."
% However, there should be an "\end{document}."
%
%===========================================================
\documentclass[11pt,twoside,titlepage]{article}
%%NEED TO ADD epsf!!
\usepackage{threeparttop}
\usepackage{graphicx}
\usepackage{latexsym}
\usepackage{color}
\usepackage{listings}
\usepackage{fancyvrb}
%\usepackage{pgf,pgfarrows,pgfnodes,pgfautomata,pgfheaps,pgfshade}
\usepackage{tikz}
\usepackage[normalem]{ulem}
\tikzset{
    %Define standard arrow tip
%    >=stealth',
    %Define style for boxes
    oval/.style={
           rectangle,
           rounded corners,
           draw=black, very thick,
           text width=6.5em,
           minimum height=2em,
           text centered},
    % Define arrow style
    arr/.style={
           ->,
           thick,
           shorten <=2pt,
           shorten >=2pt,}
}
\usepackage[noend]{algorithmic}
\usepackage[noend]{algorithm}
\newcommand{\bfor}{{\bf for\ }}
\newcommand{\bthen}{{\bf then\ }}
\newcommand{\bwhile}{{\bf while\ }}
\newcommand{\btrue}{{\bf true\ }}
\newcommand{\bfalse}{{\bf false\ }}
\newcommand{\bto}{{\bf to\ }}
\newcommand{\bdo}{{\bf do\ }}
\newcommand{\bif}{{\bf if\ }}
\newcommand{\belse}{{\bf else\ }}
\newcommand{\band}{{\bf and\ }}
\newcommand{\breturn}{{\bf return\ }}
\newcommand{\mod}{{\rm mod}}
\renewcommand{\algorithmiccomment}[1]{$\rhd$ #1}
\newenvironment{checklist}{\par\noindent\hspace{-.25in}{\bf Checklist:}\renewcommand{\labelitemi}{$\Box$}%
\begin{itemize}}{\end{itemize}}
\pagestyle{threepartheadings}
\usepackage{url}
\usepackage{wrapfig}
% removing the standard hyperref to avoid the horrible boxes
%\usepackage{hyperref}
\usepackage[hidelinks]{hyperref}
% added in the dtklogos for the bibtex formatting
\usepackage{dtklogos}
%=========================
% One-inch margins everywhere
%=========================
\setlength{\topmargin}{0in}
\setlength{\textheight}{8.5in}
\setlength{\oddsidemargin}{0in}
\setlength{\evensidemargin}{0in}
\setlength{\textwidth}{6.5in}
%===============================
%===============================
% Macro for document title:
%===============================
\newcommand{\MYTITLE}[1]%
   {\begin{center}
     \begin{center}
     \bf
     CMPSC 440\\Operating Systems\\
     Spring 2014
     \medskip
     \end{center}
     \bf
     #1
     \end{center}
}
%================================
% Macro for headings:
%================================
\newcommand{\MYHEADERS}[2]%
   {\lhead{#1}
    \rhead{#2}
    %\immediate\write16{}
    %\immediate\write16{DATE OF HANDOUT?}
    %\read16 to \dateofhandout
    \def \dateofhandout {April 7, 2014}
    \lfoot{\sc Handed out on \dateofhandout}
    %\immediate\write16{}
    %\immediate\write16{HANDOUT NUMBER?}
    %\read16 to\handoutnum
    \def \handoutnum {9}
    \rfoot{Handout \handoutnum}
   }

%================================
% Macro for bold italic:
%================================
\newcommand{\bit}[1]{{\textit{\textbf{#1}}}}

%=========================
% Non-zero paragraph skips.
%=========================
\setlength{\parskip}{1ex}

%=========================
% Create various environments:
%=========================
\newcommand{\PURPOSE}{\par\noindent\hspace{-.25in}{\bf Purpose:\ }}
\newcommand{\SUMMARY}{\par\noindent\hspace{-.25in}{\bf Summary:\ }}
\newcommand{\DETAILS}{\par\noindent\hspace{-.25in}{\bf Details:\ }}
\newcommand{\HANDIN}{\par\noindent\hspace{-.25in}{\bf Hand in:\ }}
\newcommand{\SUBHEAD}[1]{\bigskip\par\noindent\hspace{-.1in}{\sc #1}\\}
%\newenvironment{CHECKLIST}{\begin{itemize}}{\end{itemize}}


\usepackage[compact]{titlesec}

\begin{document}
\MYTITLE{Laboratory Assignment Four: Client-Server Communication with Compression}
\MYHEADERS{Laboratory Assignment Four}{Due: February 22, 2016}

\section*{Introduction}

In many distributed systems, a client is dynamically configured to communicate with a server by first downloading the
necessary code from a repository. In this laboratory assignment, you will investigate how the use of compression
influences the response time and throughput associated with these transfers of code and data from a repository to the
client. In particular, you will learn how to perform compression using at least three different algorithms (e.g., Zip,
Pack200, and Pack200+Gzip) and you will also discover how to start and monitor an HTTP server using Python.

Next, you will pick several real-world code and data repositories of compiled Java programs (i.e., they should not
contain source code) and conduct experiments to measure how compression changes the performance characteristics of a
distributed system.  Finally, using either Markdown or the \LaTeX~text formatting language, you will write a detailed
scientific report explaining the performance results that you identified while finishing this assignment. You will
complete this project by working with a partner; you and your partner will collaborate through a Slack channel and a Git
version control repository throughout the week during which you finish this assignment.

\section*{Prepare for the Assignment}

Before starting this laboratory assignment, you and your partner should read and discuss the content in Chapter 3
of your textbook. As you review this material, please make sure that you focus on the discussion of code migration in
Section 3.5, paying particularly close attention to Figure 3-17 and the content that discusses this technical diagram.
To better understand the Pack200 compression technique, you and your partner should download from the ACM Digital
Library and read ``Compressing Java Class Files'', a paper published by William Pugh in a past edition of the
Proceedings of the International Conference on Programming Language Design and Implementation. Finally, you should study
the manual pages for the {\tt jar} and {\tt pack200} programs.

\section*{Exploring Client-Server Communication with Compression}

To start this assignment, you and you partner should work together to find a ``Java Archive'', or Jar, file that
contains only Java bytecodes and other resources (e.g., images or data files) that are needed to run a Java program.
Please make sure that you are not picking Jar files that contain Java source code --- to ensure that your experiments
directly connect to the scenario depicted in Figure 3-17 the chosen archives should only contain Java bytecodes and
other types of files that would support the execution of a program. In addition to running experiments with the Jar
files that you and you partner find, I have provided two files to you that you can obtain through the ``share''
repository for this course; please type the ``{\tt git pull}'' to download these files.

What is the size of the two Jar files that I have provided to you? Can you report the size of these files in bytes? Once
you know their correct size, go ahead and learn more about how to run the {\tt pack200} command in your terminal window.
For example, if I wanted to perform both Pack200 and Gzip compression on one of the provided Jar files, I could type the
following command ``{\tt pack200 JavaRunner.pack.gz JavaRunner.jar}'' in the terminal. Now, which of the two archives
for the JavaRunner program are bigger? Can you explain why this is the case? After studying the manual page for {\tt
pack200}, can you learn how to perform Pack200 compression in a standalone fashion?

Note that you already have these files, such as {\tt JavaRunner.jar} in a Jar file format. If you want to discover the
performance of the {\tt jar} compression technique, then you will have to first decompress it and then run the command
again in compression mode. If you want to learn more about how to run the {\tt jar} command, please read its manual page
or talk with the instructor.  Since it is important to quantify the time required for compression with either the {\tt
jar} or the {\tt pack200} commands, you should also ensure that you know how to preface one of your previously executed
command-lines with either {\tt time} or {\tt /usr/bin/time}. Using one of these tools, can you determine how long it
takes to compress the code and data associated with your chosen Java programs? Finally, you and your partner should
discuss an equation to calculate the ``percentage reduction'' and then use this equation for these files. Which method
is the best at compressing the \mbox{code and data?}

So far, you have not actually transferred these archives across the network, as is depicted in Figure 3-17 of your
textbook. To accomplish this task, please change into the directory that contains all of your archives and type the
command ``{\tt python -m SimpleHTTPServer 4200}'' to start an HTTP server using the Python programming language. Now,
you can use the {\tt wget} command in another terminal window to download a compressed archive. You and your partner
should now discuss equations and means for calculating both response time and throughput. Once you have agreed on the
correct equations, you should use them to compute these metrics for some of the archives. Which compression technique
tends to yield the best values? Can you explain why? Please remember that if you want to transfer files with HTTP from
one computer in the Alden Hall network to another, then you need to use certain ports that are not blocked by the
firewall.

\section*{Empirically Studying Compression Methods}

You and your partner(s) should create a new repository called {\tt cs441S2016-lab04-<first user name>-<second use
name>}.  Next, make sure that the individuals in your partnership --- and the course instructor --- have access to the
repository that you will use for this assignment.  Students who have questions about the use of the Git version control
system should ensure that they have resolved them before leaving the laboratory session today. Finally, please make a
Slack channel, for your team and the instructor, so that you can easily communicate while completing this assignment.



\section*{Summary of the Required Deliverables}

This assignment invites you to submit printed and signed versions of the following deliverables. Additionally,
all of these deliverables must be in the repository that you created for this assignment.

\begin{enumerate}

    \item

    \item

    \item Using both text and diagrams, a description of client-server communication with compression.

    \item A detailed paper that reports on the empirical results arising from the use of the benchmarks.

    \item A description of the challenges that you encountered when completing this assignment.

    \item A description of the tasks completed by the members of your partnership.

\end{enumerate}

In adherence to the Honor Code, students should complete this assignment on an individual basis. While it is appropriate
for students in this class to have high-level conversations about the assignment, it is necessary to distinguish
carefully between the student who discusses the principles underlying a problem with others and the student who produces
assignments that are identical to, or merely variations on, someone else's work.  With the exception of the provided
source code, deliverables that are otherwise nearly identical to the work of others will be taken as evidence of
violating the \mbox{Honor Code}. This means that, for instance, all of the other comments, source code, data, and
written reports should be the original work of the two members of the partnership. Students who have questions about the
Honor Code and how it applies to this assignment should schedule a meeting with the course instructor before this
assignment's due date.

\end{document}
