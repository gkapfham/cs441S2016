%!TEX root=cs441S2016_lab4.tex

\input{labspre.tex}

\usepackage[compact]{titlesec}

\begin{document}
\MYTITLE{Laboratory Assignment Four: Client-Server Communication with Compression}
\MYHEADERS{Laboratory Assignment Four}{Due: February 22, 2016}

\section*{Introduction}

In many distributed systems, a client is dynamically configured to communicate with a server by first downloading the
necessary code from a repository. In this laboratory assignment, you will investigate how the use of compression
influences the response time and throughput associated with these transfers of code and data from a repository to the
client. In particular, you will learn how to perform compression using at least three different algorithms (e.g., Zip,
Pack200, and Pack200+Gzip) and you will also discover how to start and monitor an HTTP server using Python.

Next, you will pick several real-world code and data repositories of compiled Java programs (i.e., they should not
contain source code) and conduct experiments to measure how compression changes the performance characteristics of a
distributed system.  Finally, using either Markdown or the \LaTeX~text formatting language, you will write a detailed
scientific report explaining the performance results that you identified while finishing this assignment. You will
complete this project by working with a partner; you and your partner will collaborate through a Slack channel and a Git
version control repository throughout the week during which you finish this assignment.

\section*{Review Your Textbook}

Before starting this laboratory assignment, you and your partner should read and discuss the content in Chapter 3
of your textbook. As you review this material, please make sure that you focus on the discussion of code migration in
Section 3.5, paying particularly close attention to Figure 3-17 and the content that discusses this technical diagram.
To better understand the Pack200 compression technique, you and your partner should download from the ACM Digital
Library and read ``Compressing Java Class Files'', a paper published by William Pugh in a past edition of the
Proceedings of the International Conference on Programming Language Design and Implementation.


\section*{Exploring Client-Server Communication}



\section*{Implementing and Using a Benchmarking Framework}



\section*{Summary of the Required Deliverables}

This assignment invites you to submit printed and signed versions of the following deliverables. Additionally,
all of these deliverables must be in the repository that you created for this assignment.

\begin{enumerate}

    \item The well-commented source code of the Java classes that form the two ``useful'' benchmarks,

    \item The well-commented source code of the Java classes for the four ``baseline'' benchmarks.

    \item Using both text and diagrams, a description of client-server communication with sockets.

    \item A detailed paper that reports on the empirical results arising from the use of the benchmarks.

    \item A description of the challenges that you encountered when completing this assignment.

\end{enumerate}

In adherence to the Honor Code, students should complete this assignment on an individual basis. While it is appropriate
for students in this class to have high-level conversations about the assignment, it is necessary to distinguish
carefully between the student who discusses the principles underlying a problem with others and the student who produces
assignments that are identical to, or merely variations on, someone else's work.  With the exception of the provided
source code, deliverables that are otherwise nearly identical to the work of others will be taken as evidence of
violating the \mbox{Honor Code}. This means that, for instance, all of the other comments, source code, data, and
written reports should be the original work of the two members of the partnership. Students who have questions about the
Honor Code and how it applies to this assignment should schedule a meeting with the course instructor before this
assignment's due date.

\end{document}
