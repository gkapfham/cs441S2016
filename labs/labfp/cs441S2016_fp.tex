%!TEX root=cs440S2014-lab7.tex
% mainfile: cs440S2014-lab7.tex 
% CS 580 style
% Typical usage (all UPPERCASE items are optional):
%       \input 580pre
%       \begin{document}
%       \MYTITLE{Title of document, e.g., Lab 1\\Due ...}
%       \MYHEADERS{short title}{other running head, e.g., due date}
%       \PURPOSE{Description of purpose}
%       \SUMMARY{Very short overview of assignment}
%       \DETAILS{Detailed description}
%         \SUBHEAD{if needed} ...
%         \SUBHEAD{if needed} ...
%          ...
%       \HANDIN{What to hand in and how}
%       \begin{checklist}
%       \item ...
%       \end{checklist}
% There is no need to include a "\documentstyle."
% However, there should be an "\end{document}."
%
%===========================================================
\documentclass[11pt,twoside,titlepage]{article}
%%NEED TO ADD epsf!!
\usepackage{threeparttop}
\usepackage{graphicx}
\usepackage{latexsym}
\usepackage{color}
\usepackage{listings}
\usepackage{fancyvrb}
%\usepackage{pgf,pgfarrows,pgfnodes,pgfautomata,pgfheaps,pgfshade}
\usepackage{tikz}
\usepackage[normalem]{ulem}
\tikzset{
    %Define standard arrow tip
%    >=stealth',
    %Define style for boxes
    oval/.style={
           rectangle,
           rounded corners,
           draw=black, very thick,
           text width=6.5em,
           minimum height=2em,
           text centered},
    % Define arrow style
    arr/.style={
           ->,
           thick,
           shorten <=2pt,
           shorten >=2pt,}
}
\usepackage[noend]{algorithmic}
\usepackage[noend]{algorithm}
\newcommand{\bfor}{{\bf for\ }}
\newcommand{\bthen}{{\bf then\ }}
\newcommand{\bwhile}{{\bf while\ }}
\newcommand{\btrue}{{\bf true\ }}
\newcommand{\bfalse}{{\bf false\ }}
\newcommand{\bto}{{\bf to\ }}
\newcommand{\bdo}{{\bf do\ }}
\newcommand{\bif}{{\bf if\ }}
\newcommand{\belse}{{\bf else\ }}
\newcommand{\band}{{\bf and\ }}
\newcommand{\breturn}{{\bf return\ }}
\newcommand{\mod}{{\rm mod}}
\renewcommand{\algorithmiccomment}[1]{$\rhd$ #1}
\newenvironment{checklist}{\par\noindent\hspace{-.25in}{\bf Checklist:}\renewcommand{\labelitemi}{$\Box$}%
\begin{itemize}}{\end{itemize}}
\pagestyle{threepartheadings}
\usepackage{url}
\usepackage{wrapfig}
% removing the standard hyperref to avoid the horrible boxes
%\usepackage{hyperref}
\usepackage[hidelinks]{hyperref}
% added in the dtklogos for the bibtex formatting
\usepackage{dtklogos}
%=========================
% One-inch margins everywhere
%=========================
\setlength{\topmargin}{0in}
\setlength{\textheight}{8.5in}
\setlength{\oddsidemargin}{0in}
\setlength{\evensidemargin}{0in}
\setlength{\textwidth}{6.5in}
%===============================
%===============================
% Macro for document title:
%===============================
\newcommand{\MYTITLE}[1]%
   {\begin{center}
     \begin{center}
     \bf
     CMPSC 440\\Operating Systems\\
     Spring 2014
     \medskip
     \end{center}
     \bf
     #1
     \end{center}
}
%================================
% Macro for headings:
%================================
\newcommand{\MYHEADERS}[2]%
   {\lhead{#1}
    \rhead{#2}
    %\immediate\write16{}
    %\immediate\write16{DATE OF HANDOUT?}
    %\read16 to \dateofhandout
    \def \dateofhandout {April 7, 2014}
    \lfoot{\sc Handed out on \dateofhandout}
    %\immediate\write16{}
    %\immediate\write16{HANDOUT NUMBER?}
    %\read16 to\handoutnum
    \def \handoutnum {9}
    \rfoot{Handout \handoutnum}
   }

%================================
% Macro for bold italic:
%================================
\newcommand{\bit}[1]{{\textit{\textbf{#1}}}}

%=========================
% Non-zero paragraph skips.
%=========================
\setlength{\parskip}{1ex}

%=========================
% Create various environments:
%=========================
\newcommand{\PURPOSE}{\par\noindent\hspace{-.25in}{\bf Purpose:\ }}
\newcommand{\SUMMARY}{\par\noindent\hspace{-.25in}{\bf Summary:\ }}
\newcommand{\DETAILS}{\par\noindent\hspace{-.25in}{\bf Details:\ }}
\newcommand{\HANDIN}{\par\noindent\hspace{-.25in}{\bf Hand in:\ }}
\newcommand{\SUBHEAD}[1]{\bigskip\par\noindent\hspace{-.1in}{\sc #1}\\}
%\newenvironment{CHECKLIST}{\begin{itemize}}{\end{itemize}}


\usepackage[compact]{titlesec}

\begin{document} \MYTITLE{Final Project: Advanced Topics in Distributed Systems}
\MYHEADERS{Final Project}{Due: Tuesday, May 3, 2015 at 11:00 am}

\section*{Introduction}

Throughout the semester, you have learned more about the fundamental topics in distributed systems by studying, in a
hands-on fashion, topics such as remote communication (through sockets and remote procedure calls and remote method
invocations), distributed file systems, and performance evaluation. This final project invites you to explore, in
greater detail, an advanced topic in the field of distributed systems. Working in teams of four students each, you will
learn more about how to implement, evaluate, and/or simulate key components or facets of an distributed system. You will
also write and speak about different topics in distributed systems.  Finally, you will gain additional experience in
collaboratively working on a team to complete and present a large-scale project.

\section*{Description of the Topics}

The student-chosen teams of four students (comprised largely of different partners than before) should pick one of the
following ten projects.  Please note that a team selecting the team-designed project must first discuss their idea with
the instructor and receive feedback and then final approval. Finally, your team should be aware that, while the
instructor can assist you in resolving challenges, each of its members is ultimately responsible for ensuring the
feasibility of the proposed project.

\begin{enumerate}

  \item {\bf Simulation of Distributed Systems}: You and your team members will pick a specific component or algorithm
    in an distributed system, such as load balancer or process scheduler or synchronizer, and implement a full-featured
    simulation of it.  Then, you will use the simulator to empirically investigate the characteristics and trade-offs of
    distributed systems.

  \item {\bf Distributed File System Performance Evaluation}: You and your team members will use one or more file system
    benchmarking tools (or create your own) to evaluate the performance of different combinations of file systems and
    disks.  After deciding which benchmarking system(s) you will use, you will configure the benchmarks, run
    experiments, and report results.

  \item {\bf Benchmarking the Linux Operating System}: Since a high-performance distributed system is comprised of
    individual nodes, you and your team members will use one or more benchmarking tools for the Linux operating system
    to evaluate the performance of different aspects of a computer workstation running Linux.  After picking the aspects
    of Linux system performance that you will study and selecting the right (or, implementing your own) benchmarks, you
    will configure the system, run the experiments, and report on the results.

  \item {\bf Cloud Computing}: You and your team members will use existing cloud-computing frameworks, such as OwnCloud
    or Tonido, to install, configure, test, evaluate, document, deploy, and demonstrate a fully functional
    cloud-computing solution to be used by students and faculty in the Department of Computer Science. Although it will
    be exclusively accessible to people at Allegheny College, the completed system should be similar to existing
    systems.

  \item {\bf Cluster Computing}: Using your own computer hardware, or hardware provided by the department's systems
    administrator, you and your team will build your own cluster computer.  After installing, configuring, testing,
    documenting, and deploying your hardware and software, you should evaluate it through a series of performance
    studies and report on your results.

  \item {\bf Virtual Machines}: You and your team will investigate the installation and use of virtual machines on
    operating systems such as Windows, Linux, and Mac OSX.  Beyond downloading, installing, and configuring one or more
    virtualization environments, the team members should either pick a type of problem that they want to solve with
    virtualization or a performance characterization of virtualization that they would like to conduct. Ultimately, you
    and your team should report on both your experiences and your performance results.

  \item {\bf Distributed Heuristic Search}: You and your team will investigate the use of heuristic search techniques
    (e.g., hill climbing, simulated annealing, and genetic algorithms) and learn more about how they may incur high time
    overheads. Then, you and your team members will design, implement, and empirically evaluate a distributed (or,
    parallel) version of one or more of these algorithms that solves a real-world problem. After implementing your
    distributed heuristic search method, you should experimentally evaluate it and then report on the results.

  \item {\bf Distributed Test Suite Execution}: You and your team will learn how to use unit testing frameworks such as
    JUnit and TestNG. Then, with the understanding that many real-world test suites take a long time to run, you should
    implement and empirically evaluate a distributed (or, parallel) test execution framework. Teams that are interested
    in this project should read the blog posts ``Hard core multi-core with TestNG'' and ``More on multithreaded
    topological sorting'' written by the creator of TestNG. Finally, teams selecting this project should be prepared to
    either implement or identify and download large test suites suitable for use in the empirical study measuring the
    performance of distributed test execution.

  \item {\bf Distributed Data Analysis}: You and your team will learn how to use the data analysis libraries in the R
    language for statistical computing and then implement and empirically evaluate the efficiency of a distributed data
    analysis system. Teams that pick this project should learn more about how to perform data analysis in R using
    packages like ``dplyr'' and then explore ways to make their code faster through parallelization and/or the use of
    multiple computer nodes. If your team selects this project, please make sure that you can identify several
    real-world data sets and ultimately import them into R for subsequent analysis.

  \item {\bf Remote Communication}: You and your team members will implement, test, document, evaluate, and
    demonstrate a complete software system that performs some type of remote communication.  Teams that select this
    project should consider using techniques like sockets and/or remote procedure calls to implement a program that solves
    a challenging problem.

  \item {\bf Student-Designed Project}: You and your team will develop an idea for your own project that focuses on
    advanced topics in distributed systems. After receiving the instructor's approval for your idea, you will complete
    the project and report on your results. While it is acceptable for students to explore a topic that we already
    investigated during a laboratory assignment, teams that take this route should clearly articulate how their proposed
    project will yield a substantial extension over their prior work on a laboratory assignment.

\end{enumerate}

\vspace*{-.05in}

\section*{Final Project Deadlines}

% This assignment invites you to submit printed and signed versions of the following deliverables:

\vspace*{-.05in}
\begin{enumerate}

  \itemsep0in

  \item {\bf Final Project Assigned:} March 28, 2016

    After meeting with your team members, pick a topic for your final project.  Remember, if you select the
    student-designed project, you must first have your project verified by the course instructor.  Next, make sure that
    you create a Git repository that can be accessed by all members of the team. Finally, write and submit a
    one-paragraph description of your idea.

  \item {\bf Final Project Proposal}: April 11, 2016

    You and your team will submit a one-page proposal that describes the idea for your final project.  The proposal
    should have an informative title, an abstract, a description of the main idea, a plan for completing the work, and
    an initial listing of the roles for \mbox{each member}.

  \item {\bf Status Update}: April 18, 2016

    You and your team will submit a one-page status update that describes the finalized roles of each team member,
    explains the tasks that you have already completed, and outlines a plan for ensuring that all members can easily
    contribute to the successful finish of the final project.

  \item {\bf Final Project Demonstrations}: April 25, 2016

    Standing at the front of the classroom, you and your team will give a short demonstration, showing all of your
    system's key features, to the instructor and all of the students in the class.

  \item {\bf Final Project Presentations}: May 2, 2016

    Using slides and demonstrations as needed, you and your team will give a short ten-minute presentation of your final
    results and participate in a five minute question and answer session.

  \item {\bf Final Project Submission}: May 3, 2016

    You and your team will submit a single printed and signed version of your final project. In addition, you and your
    team will ensure that all of your project's deliverables (e.g., source code, data, team evaluations, and report) are
    available through a BitBucket repository.

\end{enumerate}

\vspace*{-.15in}

\section*{Honor Code}

The Academic Honor Program that governs the entire academic program at Allegheny College is described in the Allegheny
Course Catalogue.  The Honor Program applies to all work that is submitted for academic credit or to meet non-credit
requirements for graduation at Allegheny College.  All students who have enrolled in the College will work under the
Honor Program.  Each student who has matriculated at the College has acknowledged the following pledge:

\vspace*{-.1in}
\begin{quote}
I hereby recognize and pledge to fulfill my responsibilities, as defined in the Honor Code, and to maintain the
integrity of both myself and the College community as a whole.
\end{quote}
\vspace*{-.1in}

\noindent It is recognized that an important part of the learning process in any course, and particularly one in
computer science, derives from thoughtful discussions with teachers and fellow students.  Such dialogue is encouraged.
However, it is necessary to distinguish carefully between the team that discusses the principles underlying a problem
with others and the teams that produces assignments that are identical to, or merely variations on, someone else's
work.  While it is acceptable for teams in this class to discuss their programs, data sets, and reports with their
classmates, deliverables that are nearly identical to the work of others will be taken as evidence of violating the
\mbox{Honor Code}.

% In adherence to the honor code, students should complete this assignment while only collaborating with those students
% who are a part of their chosen team. While it is appropriate for students in this class to have high-level
% conversations about the assignment with non-team members, it is necessary to distinguish carefully between the team that
% discusses the principles underlying a problem with others and the team that produces an assignment that is identical to,
% or merely variations on, the work of someone not on the team.  As such, deliverables that are nearly identical to the
% work of others will be taken as evidence of violating the \mbox{Honor Code}.

\end{document}
