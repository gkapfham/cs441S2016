%!TEX root=cs440S2014-fp.tex 
% mainfile: cs440S2014-fp.tex 

\input{labspre.tex}

\usepackage[compact]{titlesec}

\begin{document} \MYTITLE{Final Project: Advanced Topics in Operating Systems} 
\MYHEADERS{Final Project}{Due: Thursday, May 1, 2014 at 7:00 pm}

% \vspace*{-.32in}

\section*{Introduction}

Throughout the semester, you have learned more about the basics of operating systems by studying, in a hands-on fashion,
topics such as file systems, process scheduling, and memory management.  This final project invites you to explore, in
greater detail, an advanced topic in the field of operating systems. Working in teams of four students each, you will
learn more about how to implement, evaluate, and/or simulate key components or facets of an operating system. You will
also write and speak about different topics in operating systems.  Finally, you will gain additional experience in
collaboratively working on a team to complete a large-scale project.

\section*{Description of the Topics}

The student-chosen teams of four students are invited to pick one of the following ten projects.  Please note that a
team selecting the team-designed project must first discuss their idea with the instructor and receive feedback and
then final approval.   

\begin{enumerate}

  \item {\bf Simulation of Operating Systems}: You and your team members will pick a specific component or algorithm
    in an operating system, such as the cache manager, process scheduler, or page replacement algorithm, and implement
    a full-featured simulation of it.  Then, you will use the simulator to investigate the characteristics and
    trade-offs of operating systems.

  \item {\bf File System Performance Evaluation}: You and your team members will use one or more file system
    benchmarking tools to evaluate the performance of different combinations of file systems and disks.  After deciding
    which benchmarking systems you will use, you will configure the benchmarks, run experiments, and report on the
    results.

  \item {\bf Benchmarking the Linux Operating System}: You and your team members will use one or more benchmarking tools
    for the Linux operating system to evaluate the performance of different aspects of a computer workstation running
    Linux.  After picking the aspects of Linux system performance that you will study and selecting the right
    benchmarks, you will configure the system, run the experiments, and report on the results.

  \item {\bf Remote Communication}: You and your team members will implement, test, document, evaluate, and
    demonstrate a complete software system that performs some type of remote communication.  Teams that select this
    project should consider using techniques like sockets and remote procedure calls to implement a program that solves
    a challenging problem.

  \item {\bf Cloud Computing}: You and your team members will use existing cloud-computing frameworks, such as OwnCloud,
    to install, configure, test, evaluate, document, deploy, and demonstrate a fully functional cloud-computing solution
    to be used by students and faculty in the Department of Computer Science. Although it will be exclusively accessible
    to people at Allegheny College, the completed system should be similar to Ubuntu One.

  \item {\bf Cluster Computing}: Using your own computer hardware, or hardware provided by the department's systems
    administrator, you and your team will build your own cluster computer.  After installing, configuring, testing,
    documenting, and deploying your hardware and software, you should evaluate it through a series of performance
    studies and report on your results.

  \item {\bf Virtual Machines}: You and your team will investigate the installation and use of virtual machines on
    operating systems such as Windows, Linux, and Mac OSX.  Beyond downloading, installing, and configuring one or more
    virtualization environments, the team members should either pick a type of problem that they want to solve with
    virtualization or a performance characterization of virtualization that they would like to conduct.  

  \item {\bf Student-Designed Project}: You and your team will develop an idea for your own project that focuses on
    advanced topics in operating systems. After receiving the instructor's approval for your idea, you will complete the
    project and report on your results.
    
\end{enumerate} 



\section*{Final Project Deadlines}

This assignment invites you to submit printed and signed versions of the following deliverables: 

\vspace*{-.05in}
\begin{enumerate}

  \itemsep0in

  \item {\bf Project Assigned:} April 7, 2014

    After meeting with your team members, pick a topic for your final project.  Remember, if you select the
    student-designed project, you must first have your project verified by the course instructor.  Next, make sure that
    you create a Git repository that can be accessed by all members of the team. Finally, write and submit a
    one-paragraph description of your idea.

  \item {\bf Project Proposal}: April 14, 2014

    You and your team will submit a one-page proposal that describes the idea for your final project.  The proposal
    should have an informative title, an abstract, a description of the main idea, a plan for completing the work, and
    an initial listing of the roles for \mbox{each member}. 

  \item {\bf Status Update}: April 21, 2014

    You and your team will submit a one-page status update that describes the finalized roles of each team member,
    explains the tasks that you have already completed, and outlines a plan for ensuring that all members can easily
    contribute to the successful finish of the final project.

  \item {\bf Project Presentations}: April 28, 2014

    Using slides and demonstrations as needed, you and your team will give a short ten-minute presentation of your final
    results and participate in a five minute question and answer session.

\end{enumerate}
\vspace*{-.05in}

In adherence to the honor code, students should complete this assignment while only collaborating with those students
who are a part of their chosen team. While it is appropriate for students in this class to have high-level
conversations about the assignment with non-team members, it is necessary to distinguish carefully between the team that
discusses the principles underlying a problem with others and the team that produces an assignment that is identical to,
or merely variations on, the work of someone not on the team.  As such, deliverables that are nearly identical to the
work of others will be taken as evidence of violating the \mbox{Honor Code}.  

  \end{document}
