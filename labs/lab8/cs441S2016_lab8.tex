\input{labspre.tex}

\usepackage[compact]{titlesec}

\begin{document}
\MYTITLE{Laboratory Assignment Eight: Benchmarking Distributed and Local File Systems}
\MYHEADERS{Laboratory Assignment Eight}{Due: March 14, 2016}

\section*{Introduction}

In the Java programming language, it is common to implement a distributed object-based system using the remote method
invocation (RMI) classes. In this laboratory assignment, you will download and use a Java system that is derived from
the example provided at \url{https://github.com/pothoven/simple-rmi}. Once you have correctly compiled and configured
this system, you will run it and explore all of the features that it provides. Next, you will conduct an experiment to
measure the response time associated with the file upload feature that the {\tt simple-rmi} system furnishes. Finally,
you will add one new feature to the provided system. In addition to running these experiments and adding a new feature,
you will write a report, using either Markdown or \LaTeX, that presents on your performance results and reflects on your
implementation experiences.

\section*{Review Your Textbook}

Before starting this assignment, you should review Chapters 1 through 4 of your textbook. If you have a question about
this content, then please resolve it before starting this laboratory assignment. Next, you should examine the material
in Sections 10.1 and 10.2 that discusses the features commonly found in a distributed object-based system. Additionally,
students should pay close attention to the paragraphs in Section 10.3 that focus on either remote method invocation in the Java
programming language or the Java distributed-object model. Although it is not the emphasis of this assignment, students
who want to learn about security protocols and how they are used in distributed object-based systems should study the
content in Section 10.8 and the security policy file in the provided implementation.  Please see the instructor if you
have any questions.

\section*{Exploring Remote Method Invocation in Java}

Since you configured your connection to the ``share'' repository for this course in a previous laboratory assignment,
you should be able to change into the {\tt cs441S2016-share/} directory and type the ``{\tt git pull}'' command to gain
access to the many Java classes needed to complete this assignment. Please recall that the source code that you received
for this laboratory assignment is derived from the Java system that is available at
\url{https://github.com/pothoven/simple-rmi}; students who would like to learn more about this system are encouraged to
visit the referenced GitHub page.

You will notice that this project furnishes a {\tt build.xml} file that is responsible for compiling the system and then
preparing it for use through a Java command-line interface. If you would like to compile the system, then you need to
type the command ``{\tt ant compile}'' in your terminal window. Please note that the deprecation message that you
received after running this command is acceptable for this assignment; the current configuration of the {\tt build.xml}
file creates a distributed object-based system that is the closest to the one that we have discussed in our past class
meetings and thus is appropriate for supporting further learning and discussion during today's laboratory session.
Please see the course instructor if you could not compile the system with {\tt ant}.

Once you have compiled the system, you should go into the {\tt bin/} directory and see what bytecode files have been
produced. Do you see any files that are specifically needed for a distributed system? If you look into the main
directory of this system, you should also notice that it contains a {\tt Simple.properties} file. Before you run the
next command, please make sure that this file is also placed in the {\tt bin/} directory. Now, you can type the command
``{\tt ant jar}'' in your terminal window. At this point, you are ready to run the client and the server according to
the commands that are provided in the documentation. What commands did you type? What output did they produce? Please
ensure that you try all of the commands supported by this distributed object-based system.

When you are using the ``upload'' functionality, you are advised to specify a name for the ``destination'' file.
Additionally, you should add in timing code so that you can perform an experiment to measure the time overheads
associated with performing uploads. Next, you should review and comment all of the provided Java source code.  Can you
determine the port used by the ``RMI registry'' that this system starts? Can you find the class that calls {\tt rebind}?
What is the purpose of this method? Finally, you should add one new feature to this distributed object-based system.

\section*{Applications and Enhancements of Java RMI}

Java RMI serves as the ``backbone'' for a wide variety of different types of distributed systems. Additionally, there
are many ways in which researchers and practitioners have extended Java RMI so that it provides new features that are
not a part of the standard distribution. During the final portion of this laboratory assignment, you should search the
ACM Digital Library --- available from \url{http://dl.acm.org/} --- and download and read two papers. After carefully
studying the papers, you should write a short one or two paragraph review of each paper that highlights its key
contributions. Finally, your review of the paper should comment on how the implementation that you used in this
assignment is similar to and different from the one presented in the paper.

\section*{Summary of the Required Deliverables}

This assignment invites you to submit printed and signed versions of these deliverables. All of these deliverables must
also be in a {\tt cs441S2016-<your user name>} repository created for \mbox{this course}.

\vspace*{-.1in}

\begin{enumerate}
  \itemsep 0em

  \item The well-commented source code of all of the Java classes in the final distributed system.

  \item Using both text and diagrams, a description of client-server communication with Java RMI.

  \item A document that summarizes two published papers that report on the use of Java RMI.

  \item A paper that responds to the other questions that this assignment poses about Java RMI.

  \item A reflection on the challenges that you encountered when completing this assignment.

\end{enumerate}

\vspace*{-.1in}

% With the exception of the provided source code, deliverables that are otherwise nearly identical to the work of others
% will be taken as evidence of violating the \mbox{Honor Code}.

In adherence to the Honor Code, students should complete this assignment on an individual basis. While it is appropriate
for students in this class to have high-level conversations about the assignment, it is necessary to distinguish
carefully between the student who discusses the principles underlying a problem with others and the student who produces
assignments that are identical to, or merely variations on, someone else's work.  With the exception of the provided
source code, deliverables that are otherwise nearly identical to the work of others will be taken as evidence of
violating the \mbox{Honor Code}. This means that, for instance, all of the other comments, source code, data, and
written reports should be the original work of the student completing this assignment.

% Students who have questions about the Honor Code and how it applies to this laboratory assignment should schedule a
% meeting with the course instructor before this assignment's due date.

\end{document}
