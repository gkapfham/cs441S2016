\input{labspre.tex}

\usepackage[compact]{titlesec}

\begin{document}
\MYTITLE{Laboratory Assignment Eight: Benchmarking Distributed and Local File Systems}
\MYHEADERS{Laboratory Assignment Eight}{Due: March 30, 2016}

\section*{Introduction}



\section*{Review Your Textbook}

Before starting this assignment, you should review Chapters 1 through 4 of your textbook. If you have a question about
this content, then please resolve it before starting this laboratory assignment. Next, you should examine the material
in Sections 10.1 and 10.2 that discusses the features commonly found in a distributed object-based system. Additionally,
students should pay close attention to the paragraphs in Section 10.3 that focus on either remote method invocation in the Java
programming language or the Java distributed-object model. Although it is not the emphasis of this assignment, students
who want to learn about security protocols and how they are used in distributed object-based systems should study the
content in Section 10.8 and the security policy file in the provided implementation.  Please see the instructor if you
have any questions.

\section*{Leverage Past Experience in File System Benchmarking}



\section*{Judiciously Use a File System Benchmarking Tool}


\section*{Summary of the Required Deliverables}

This assignment invites you to submit printed and signed versions of these deliverables. All of these deliverables must
also be in a {\tt cs441S2016-<your user name>} repository created for \mbox{this course}.

\vspace*{-.1in}

\begin{enumerate}
  \itemsep 0em

  \item The well-commented source code of all of the Java classes in the final distributed system.

  \item Using both text and diagrams, a description of client-server communication with Java RMI.

  \item A document that summarizes two published papers that report on the use of Java RMI.

  \item A paper that responds to the other questions that this assignment poses about Java RMI.

  \item A reflection on the challenges that you encountered when completing this assignment.

\end{enumerate}

\vspace*{-.1in}

% With the exception of the provided source code, deliverables that are otherwise nearly identical to the work of others
% will be taken as evidence of violating the \mbox{Honor Code}.

In adherence to the Honor Code, students should complete this assignment on an individual basis. While it is appropriate
for students in this class to have high-level conversations about the assignment, it is necessary to distinguish
carefully between the student who discusses the principles underlying a problem with others and the student who produces
assignments that are identical to, or merely variations on, someone else's work.  With the exception of the provided
source code, deliverables that are otherwise nearly identical to the work of others will be taken as evidence of
violating the \mbox{Honor Code}. This means that, for instance, all of the other comments, source code, data, and
written reports should be the original work of the student completing this assignment.

% Students who have questions about the Honor Code and how it applies to this laboratory assignment should schedule a
% meeting with the course instructor before this assignment's due date.

\end{document}
