%!TEX root=cs440S2014-lab7.tex
% mainfile: cs440S2014-lab7.tex 
% CS 580 style
% Typical usage (all UPPERCASE items are optional):
%       \input 580pre
%       \begin{document}
%       \MYTITLE{Title of document, e.g., Lab 1\\Due ...}
%       \MYHEADERS{short title}{other running head, e.g., due date}
%       \PURPOSE{Description of purpose}
%       \SUMMARY{Very short overview of assignment}
%       \DETAILS{Detailed description}
%         \SUBHEAD{if needed} ...
%         \SUBHEAD{if needed} ...
%          ...
%       \HANDIN{What to hand in and how}
%       \begin{checklist}
%       \item ...
%       \end{checklist}
% There is no need to include a "\documentstyle."
% However, there should be an "\end{document}."
%
%===========================================================
\documentclass[11pt,twoside,titlepage]{article}
%%NEED TO ADD epsf!!
\usepackage{threeparttop}
\usepackage{graphicx}
\usepackage{latexsym}
\usepackage{color}
\usepackage{listings}
\usepackage{fancyvrb}
%\usepackage{pgf,pgfarrows,pgfnodes,pgfautomata,pgfheaps,pgfshade}
\usepackage{tikz}
\usepackage[normalem]{ulem}
\tikzset{
    %Define standard arrow tip
%    >=stealth',
    %Define style for boxes
    oval/.style={
           rectangle,
           rounded corners,
           draw=black, very thick,
           text width=6.5em,
           minimum height=2em,
           text centered},
    % Define arrow style
    arr/.style={
           ->,
           thick,
           shorten <=2pt,
           shorten >=2pt,}
}
\usepackage[noend]{algorithmic}
\usepackage[noend]{algorithm}
\newcommand{\bfor}{{\bf for\ }}
\newcommand{\bthen}{{\bf then\ }}
\newcommand{\bwhile}{{\bf while\ }}
\newcommand{\btrue}{{\bf true\ }}
\newcommand{\bfalse}{{\bf false\ }}
\newcommand{\bto}{{\bf to\ }}
\newcommand{\bdo}{{\bf do\ }}
\newcommand{\bif}{{\bf if\ }}
\newcommand{\belse}{{\bf else\ }}
\newcommand{\band}{{\bf and\ }}
\newcommand{\breturn}{{\bf return\ }}
\newcommand{\mod}{{\rm mod}}
\renewcommand{\algorithmiccomment}[1]{$\rhd$ #1}
\newenvironment{checklist}{\par\noindent\hspace{-.25in}{\bf Checklist:}\renewcommand{\labelitemi}{$\Box$}%
\begin{itemize}}{\end{itemize}}
\pagestyle{threepartheadings}
\usepackage{url}
\usepackage{wrapfig}
% removing the standard hyperref to avoid the horrible boxes
%\usepackage{hyperref}
\usepackage[hidelinks]{hyperref}
% added in the dtklogos for the bibtex formatting
\usepackage{dtklogos}
%=========================
% One-inch margins everywhere
%=========================
\setlength{\topmargin}{0in}
\setlength{\textheight}{8.5in}
\setlength{\oddsidemargin}{0in}
\setlength{\evensidemargin}{0in}
\setlength{\textwidth}{6.5in}
%===============================
%===============================
% Macro for document title:
%===============================
\newcommand{\MYTITLE}[1]%
   {\begin{center}
     \begin{center}
     \bf
     CMPSC 440\\Operating Systems\\
     Spring 2014
     \medskip
     \end{center}
     \bf
     #1
     \end{center}
}
%================================
% Macro for headings:
%================================
\newcommand{\MYHEADERS}[2]%
   {\lhead{#1}
    \rhead{#2}
    %\immediate\write16{}
    %\immediate\write16{DATE OF HANDOUT?}
    %\read16 to \dateofhandout
    \def \dateofhandout {April 7, 2014}
    \lfoot{\sc Handed out on \dateofhandout}
    %\immediate\write16{}
    %\immediate\write16{HANDOUT NUMBER?}
    %\read16 to\handoutnum
    \def \handoutnum {9}
    \rfoot{Handout \handoutnum}
   }

%================================
% Macro for bold italic:
%================================
\newcommand{\bit}[1]{{\textit{\textbf{#1}}}}

%=========================
% Non-zero paragraph skips.
%=========================
\setlength{\parskip}{1ex}

%=========================
% Create various environments:
%=========================
\newcommand{\PURPOSE}{\par\noindent\hspace{-.25in}{\bf Purpose:\ }}
\newcommand{\SUMMARY}{\par\noindent\hspace{-.25in}{\bf Summary:\ }}
\newcommand{\DETAILS}{\par\noindent\hspace{-.25in}{\bf Details:\ }}
\newcommand{\HANDIN}{\par\noindent\hspace{-.25in}{\bf Hand in:\ }}
\newcommand{\SUBHEAD}[1]{\bigskip\par\noindent\hspace{-.1in}{\sc #1}\\}
%\newenvironment{CHECKLIST}{\begin{itemize}}{\end{itemize}}


\usepackage[compact]{titlesec}
\usepackage{hologo}

\begin{document}
\MYTITLE{Laboratory Assignment Eight: Benchmarking Distributed and Local File Systems}
\MYHEADERS{Laboratory Assignment Eight}{Due: March 30, 2016}

\section*{Introduction}

As noted by Tanenbaum and Van Steen, ``distributed file systems form the basis for many distributed applications.'' In
this laboratory assignment, you will create and run your own file system benchmark on both a local and a distributed
file system. Then, you will analyze the results arising from multiple runs of your benchmark to assess the relative
performance trade-offs associated with the use of local and distributed file systems. Next, you will use an established
benchmarking program to conduct further experiments to evaluate the performance of the same file systems. Before
conducting your benchmarks, you will carefully read a journal paper, published in {\em ACM Transactions on Storage},
that reports on the challenges and best practices in the field of file system and storage benchmarking. Additionally,
you will learn more about how leading experts in the field suggest that you evaluate the performance of a file system.
Ultimately, you will develop a full-featured understanding of the trade-offs related to both file systems and their
benchmarking.

\section*{Review Your Textbook}

Before starting this assignment, you should review Chapters 1 through 4 of your textbook. Since this laboratory
assignment builds on the fundamentals developed in these chapters, you should resolve question about this content before
starting this laboratory assignment. Next, you should examine the material in Sections 11.1 through 11.4 that discusses
the issues related to the design, implementation, and use of a distributed file system like the Network File System
(NFS). Students should pay close attention to the content in Section 10.1 that introduces the basic NFS architecture for
a Unix-based system. Although it is not this assignment's emphasis, students who want to learn more about topics such as
consistency, replication, fault tolerance, and security in distributed file systems should review Sections 11.5 through
11.8.  Please see the instructor if you have questions.

\section*{Leverage Past Experience in File System Benchmarking}

The authors of the paper ``A Nine Year Study of File System and Storage Benchmarking'', published in {\em ACM
Transactions on Storage}, make the following statement in the abstract of their paper: ``We found that most popular
benchmarks are flawed and many research papers do not provide clear indication of true performance.'' To learn more
about the claims made in this paper written by Traeger et al.\ please go to the ACM Digital Library and download its PDF
and, to ensure that you can cite this paper in your final report, its \hologo{BibTeX} reference. Students should
carefully read this entire paper before implementing and running all of their file system benchmarks; please see the
course instructor if you have questions about this paper. To start, it is worth noting that this paper develops two
underlying themes, which the authors summarize in the following fashion:

\begin{enumerate}
  \itemsep 0in
  \item ``Explain what was done in as much detail as possible.''
  \item ``In addition to saying what was done, say why it was done that way.''
\end{enumerate}

Additionally, the authors suggest that there are three main types of benchmarks. As you are completing this laboratory
assignment, please make sure that you are always following the two main suggestions of this paper and that you are aware
of the type of benchmark that you are either creating or running. When you are creating your own benchmark or running an
existing tool to perform benchmarking, you should first think about the research question that you are attempting to
answer. Then, you need to clearly state a hypothesis for this research question and, finally, define the
metrics that will enable you to evaluate the hypothesis and respond to the question.

\section*{Create and Use Your Own File System Benchmark}

\section*{Judiciously Use a File System Benchmarking Tool}


\section*{Summary of the Required Deliverables}

This assignment invites you to submit printed and signed versions of these deliverables. All of these deliverables must
also be in a {\tt cs441S2016-<your user name>} repository created for \mbox{this course}.

\vspace*{-.1in}

\begin{enumerate}
  \itemsep 0em

  \item The well-commented source code of all of the Java classes in the final distributed system.

  \item Using both text and diagrams, a description of client-server communication with Java RMI.

  \item A document that summarizes two published papers that report on the use of Java RMI.

  \item A paper that responds to the other questions that this assignment poses about Java RMI.

  \item A reflection on the challenges that you encountered when completing this assignment.

\end{enumerate}

\vspace*{-.1in}

% With the exception of the provided source code, deliverables that are otherwise nearly identical to the work of others
% will be taken as evidence of violating the \mbox{Honor Code}.

In adherence to the Honor Code, students should complete this assignment on an individual basis. While it is appropriate
for students in this class to have high-level conversations about the assignment, it is necessary to distinguish
carefully between the student who discusses the principles underlying a problem with others and the student who produces
assignments that are identical to, or merely variations on, someone else's work.  With the exception of the provided
source code, deliverables that are otherwise nearly identical to the work of others will be taken as evidence of
violating the \mbox{Honor Code}. This means that, for instance, all of the other comments, source code, data, and
written reports should be the original work of the student completing this assignment.

% Students who have questions about the Honor Code and how it applies to this laboratory assignment should schedule a
% meeting with the course instructor before this assignment's due date.

\end{document}
