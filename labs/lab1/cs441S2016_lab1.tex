%!TEX root=cs441S2016_lab1.tex

%!TEX root=cs440S2014-lab7.tex
% mainfile: cs440S2014-lab7.tex 
% CS 580 style
% Typical usage (all UPPERCASE items are optional):
%       \input 580pre
%       \begin{document}
%       \MYTITLE{Title of document, e.g., Lab 1\\Due ...}
%       \MYHEADERS{short title}{other running head, e.g., due date}
%       \PURPOSE{Description of purpose}
%       \SUMMARY{Very short overview of assignment}
%       \DETAILS{Detailed description}
%         \SUBHEAD{if needed} ...
%         \SUBHEAD{if needed} ...
%          ...
%       \HANDIN{What to hand in and how}
%       \begin{checklist}
%       \item ...
%       \end{checklist}
% There is no need to include a "\documentstyle."
% However, there should be an "\end{document}."
%
%===========================================================
\documentclass[11pt,twoside,titlepage]{article}
%%NEED TO ADD epsf!!
\usepackage{threeparttop}
\usepackage{graphicx}
\usepackage{latexsym}
\usepackage{color}
\usepackage{listings}
\usepackage{fancyvrb}
%\usepackage{pgf,pgfarrows,pgfnodes,pgfautomata,pgfheaps,pgfshade}
\usepackage{tikz}
\usepackage[normalem]{ulem}
\tikzset{
    %Define standard arrow tip
%    >=stealth',
    %Define style for boxes
    oval/.style={
           rectangle,
           rounded corners,
           draw=black, very thick,
           text width=6.5em,
           minimum height=2em,
           text centered},
    % Define arrow style
    arr/.style={
           ->,
           thick,
           shorten <=2pt,
           shorten >=2pt,}
}
\usepackage[noend]{algorithmic}
\usepackage[noend]{algorithm}
\newcommand{\bfor}{{\bf for\ }}
\newcommand{\bthen}{{\bf then\ }}
\newcommand{\bwhile}{{\bf while\ }}
\newcommand{\btrue}{{\bf true\ }}
\newcommand{\bfalse}{{\bf false\ }}
\newcommand{\bto}{{\bf to\ }}
\newcommand{\bdo}{{\bf do\ }}
\newcommand{\bif}{{\bf if\ }}
\newcommand{\belse}{{\bf else\ }}
\newcommand{\band}{{\bf and\ }}
\newcommand{\breturn}{{\bf return\ }}
\newcommand{\mod}{{\rm mod}}
\renewcommand{\algorithmiccomment}[1]{$\rhd$ #1}
\newenvironment{checklist}{\par\noindent\hspace{-.25in}{\bf Checklist:}\renewcommand{\labelitemi}{$\Box$}%
\begin{itemize}}{\end{itemize}}
\pagestyle{threepartheadings}
\usepackage{url}
\usepackage{wrapfig}
% removing the standard hyperref to avoid the horrible boxes
%\usepackage{hyperref}
\usepackage[hidelinks]{hyperref}
% added in the dtklogos for the bibtex formatting
\usepackage{dtklogos}
%=========================
% One-inch margins everywhere
%=========================
\setlength{\topmargin}{0in}
\setlength{\textheight}{8.5in}
\setlength{\oddsidemargin}{0in}
\setlength{\evensidemargin}{0in}
\setlength{\textwidth}{6.5in}
%===============================
%===============================
% Macro for document title:
%===============================
\newcommand{\MYTITLE}[1]%
   {\begin{center}
     \begin{center}
     \bf
     CMPSC 440\\Operating Systems\\
     Spring 2014
     \medskip
     \end{center}
     \bf
     #1
     \end{center}
}
%================================
% Macro for headings:
%================================
\newcommand{\MYHEADERS}[2]%
   {\lhead{#1}
    \rhead{#2}
    %\immediate\write16{}
    %\immediate\write16{DATE OF HANDOUT?}
    %\read16 to \dateofhandout
    \def \dateofhandout {April 7, 2014}
    \lfoot{\sc Handed out on \dateofhandout}
    %\immediate\write16{}
    %\immediate\write16{HANDOUT NUMBER?}
    %\read16 to\handoutnum
    \def \handoutnum {9}
    \rfoot{Handout \handoutnum}
   }

%================================
% Macro for bold italic:
%================================
\newcommand{\bit}[1]{{\textit{\textbf{#1}}}}

%=========================
% Non-zero paragraph skips.
%=========================
\setlength{\parskip}{1ex}

%=========================
% Create various environments:
%=========================
\newcommand{\PURPOSE}{\par\noindent\hspace{-.25in}{\bf Purpose:\ }}
\newcommand{\SUMMARY}{\par\noindent\hspace{-.25in}{\bf Summary:\ }}
\newcommand{\DETAILS}{\par\noindent\hspace{-.25in}{\bf Details:\ }}
\newcommand{\HANDIN}{\par\noindent\hspace{-.25in}{\bf Hand in:\ }}
\newcommand{\SUBHEAD}[1]{\bigskip\par\noindent\hspace{-.1in}{\sc #1}\\}
%\newenvironment{CHECKLIST}{\begin{itemize}}{\end{itemize}}


\usepackage[compact]{titlesec}

\begin{document}
\MYTITLE{Laboratory Assignment One: Using Sockets to Explore Client-Server Communication}
\MYHEADERS{Laboratory Assignment One}{Due: February 1, 2016}

\section*{Introduction}

The ability to implement and evaluate software that supports communication between client and a server is a crucial
laboratory skill that will support further work in the field of distributed systems. In this laboratory assignment, you
will use two programs that perform client-server communication through the use of a ``socket'' in the Java programming
language. Additionally, you will implement three programs that will support the experimental evaluation of the
performance of using sockets for client-server communication on a single host. Finally, using either Markdown or the
\LaTeX~text formatting languages, you will write a detailed scientific report explaining the performance results that
you identified during the completion of this assignment. You will complete this project by working with a partner; you
and your partner will collaborate through a dedicated Slack channel and a Git version control repository throughout the
week during which you finish this assignment.

\section*{Review Your Textbook}






\section*{Summary of the Required Deliverables}

This assignment invites you to submit printed and signed versions of the following deliverables:

\begin{enumerate}

    \item Using screenshots and concrete examples, a description of the key features provided by zsh

    \item A commentary on the features that zsh provides that your previous shell did not

    \item A tutorial that explains how to download, install, and configure oh-my-zsh, including how to:

    \item A screenshot showing the final configuration of your zsh prompt in all relevant contexts

    \item A description of the challenges that you encountered when customizing and using zsh

\end{enumerate}

It is recognized that not all of the students in the class may be familiar with the Git version control system and thus have some
difficulty using and configuring the git plugin for Zsh.  Students who do not have access to a Git repository should see the
instructor so that one can be made available to them for the purposes of completing this assignment. However, please note that you
are not required to extensively use a Git repository to complete this assignment. Finally, students are strongly encouraged to
write their laboratory report in \LaTeX.

In adherence to the honor code, students should complete this assignment on an individual basis. While it is appropriate for
students in this class to have high-level conversations about the assignment, it is necessary to distinguish carefully between the
student who discusses the principles underlying a problem with others and the student who produces assignments that are identical
to, or merely variations on, someone else's work.  As such, deliverables that are nearly identical to the work of others will be
taken as evidence of violating the \mbox{Honor Code}.

\end{document}
