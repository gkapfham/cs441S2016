%!TEX root=cs441S2016_lab1.tex

%!TEX root=cs440S2014-lab7.tex
% mainfile: cs440S2014-lab7.tex 
% CS 580 style
% Typical usage (all UPPERCASE items are optional):
%       \input 580pre
%       \begin{document}
%       \MYTITLE{Title of document, e.g., Lab 1\\Due ...}
%       \MYHEADERS{short title}{other running head, e.g., due date}
%       \PURPOSE{Description of purpose}
%       \SUMMARY{Very short overview of assignment}
%       \DETAILS{Detailed description}
%         \SUBHEAD{if needed} ...
%         \SUBHEAD{if needed} ...
%          ...
%       \HANDIN{What to hand in and how}
%       \begin{checklist}
%       \item ...
%       \end{checklist}
% There is no need to include a "\documentstyle."
% However, there should be an "\end{document}."
%
%===========================================================
\documentclass[11pt,twoside,titlepage]{article}
%%NEED TO ADD epsf!!
\usepackage{threeparttop}
\usepackage{graphicx}
\usepackage{latexsym}
\usepackage{color}
\usepackage{listings}
\usepackage{fancyvrb}
%\usepackage{pgf,pgfarrows,pgfnodes,pgfautomata,pgfheaps,pgfshade}
\usepackage{tikz}
\usepackage[normalem]{ulem}
\tikzset{
    %Define standard arrow tip
%    >=stealth',
    %Define style for boxes
    oval/.style={
           rectangle,
           rounded corners,
           draw=black, very thick,
           text width=6.5em,
           minimum height=2em,
           text centered},
    % Define arrow style
    arr/.style={
           ->,
           thick,
           shorten <=2pt,
           shorten >=2pt,}
}
\usepackage[noend]{algorithmic}
\usepackage[noend]{algorithm}
\newcommand{\bfor}{{\bf for\ }}
\newcommand{\bthen}{{\bf then\ }}
\newcommand{\bwhile}{{\bf while\ }}
\newcommand{\btrue}{{\bf true\ }}
\newcommand{\bfalse}{{\bf false\ }}
\newcommand{\bto}{{\bf to\ }}
\newcommand{\bdo}{{\bf do\ }}
\newcommand{\bif}{{\bf if\ }}
\newcommand{\belse}{{\bf else\ }}
\newcommand{\band}{{\bf and\ }}
\newcommand{\breturn}{{\bf return\ }}
\newcommand{\mod}{{\rm mod}}
\renewcommand{\algorithmiccomment}[1]{$\rhd$ #1}
\newenvironment{checklist}{\par\noindent\hspace{-.25in}{\bf Checklist:}\renewcommand{\labelitemi}{$\Box$}%
\begin{itemize}}{\end{itemize}}
\pagestyle{threepartheadings}
\usepackage{url}
\usepackage{wrapfig}
% removing the standard hyperref to avoid the horrible boxes
%\usepackage{hyperref}
\usepackage[hidelinks]{hyperref}
% added in the dtklogos for the bibtex formatting
\usepackage{dtklogos}
%=========================
% One-inch margins everywhere
%=========================
\setlength{\topmargin}{0in}
\setlength{\textheight}{8.5in}
\setlength{\oddsidemargin}{0in}
\setlength{\evensidemargin}{0in}
\setlength{\textwidth}{6.5in}
%===============================
%===============================
% Macro for document title:
%===============================
\newcommand{\MYTITLE}[1]%
   {\begin{center}
     \begin{center}
     \bf
     CMPSC 440\\Operating Systems\\
     Spring 2014
     \medskip
     \end{center}
     \bf
     #1
     \end{center}
}
%================================
% Macro for headings:
%================================
\newcommand{\MYHEADERS}[2]%
   {\lhead{#1}
    \rhead{#2}
    %\immediate\write16{}
    %\immediate\write16{DATE OF HANDOUT?}
    %\read16 to \dateofhandout
    \def \dateofhandout {April 7, 2014}
    \lfoot{\sc Handed out on \dateofhandout}
    %\immediate\write16{}
    %\immediate\write16{HANDOUT NUMBER?}
    %\read16 to\handoutnum
    \def \handoutnum {9}
    \rfoot{Handout \handoutnum}
   }

%================================
% Macro for bold italic:
%================================
\newcommand{\bit}[1]{{\textit{\textbf{#1}}}}

%=========================
% Non-zero paragraph skips.
%=========================
\setlength{\parskip}{1ex}

%=========================
% Create various environments:
%=========================
\newcommand{\PURPOSE}{\par\noindent\hspace{-.25in}{\bf Purpose:\ }}
\newcommand{\SUMMARY}{\par\noindent\hspace{-.25in}{\bf Summary:\ }}
\newcommand{\DETAILS}{\par\noindent\hspace{-.25in}{\bf Details:\ }}
\newcommand{\HANDIN}{\par\noindent\hspace{-.25in}{\bf Hand in:\ }}
\newcommand{\SUBHEAD}[1]{\bigskip\par\noindent\hspace{-.1in}{\sc #1}\\}
%\newenvironment{CHECKLIST}{\begin{itemize}}{\end{itemize}}


\usepackage[compact]{titlesec}

\begin{document}
\MYTITLE{Laboratory Assignment One: Using Sockets to Explore Client-Server Communication}
\MYHEADERS{Laboratory Assignment One}{Due: February 1, 2016}

\section*{Introduction}

The ability to implement and evaluate software that supports communication between a client and a server is a crucial
laboratory skill that will support further work in the field of distributed systems. In this laboratory assignment, you
will use several programs that perform client-server communication through the use of a ``socket'' in the Java
programming language. Additionally, you will implement three programs that will support the experimental evaluation of
the performance of using sockets for client-server communication on a single host. Finally, using either Markdown or the
\LaTeX~text formatting language, you will write a detailed scientific report explaining the performance results that you
identified while finishing this assignment. You will complete this project by working with a partner; you and your
partner will collaborate through a dedicated Slack channel and a Git version control repository throughout the week
during which you finish this assignment.

\section*{Review Your Textbook}

Before starting this laboratory assignment, you and your partner should read and discuss the content in Chapters 1 and 2
of your textbook. As you review this material, please make sure that you focus on the discussion of different types of
communication middleware on page 24. Also, you and your partner should study all of the material in Section 2,2, paying
particularly close attention to Figure 2-3 and the description of the interactions between a client and a server.

\section*{Access the Course Repository}

Throughout this semester, the course instructor will deliver code to the students in this class through a Git
repository, hosted by BitBucket.  During this assignment and subsequent assignments, we will securely communicate with
the Bitbucket.org servers that will host all of our projects.  In this assignment, we will perform all of the steps to
configure the accounts on the departmental servers and the Bitbucket service.  Throughout the assignment, you should
refer to the following Web site for additional information:
\url{https://confluence.atlassian.com/display/BITBUCKET/Bitbucket+101}.  As you will be required to use Git in the
remaining laboratory assignments and during the class sessions, please be sure to keep a record of all of
the steps that you complete and the challenges that you face.  You are also responsible for communicating with a partner
to ensure that each of you is able to successfully complete each of the steps outlined in this assignment.

\begin{enumerate}

  \item If you do not already have a Bitbucket account, then please go to the Bitbucket Web site and create one --- make sure
    that you use your {\tt allegheny.edu} email address so that you can create an unlimited number of free Bitbucket
    repositories while you are a student.

  \item If you have never done so before, you must use the {\tt ssh-keygen} program to create secure-shell keys that you
    can use to support your communication with the Bitbucket servers. Follow the prompts to create your keys and save
    them in the default directory (press ``Enter'' after you are prompted: ``{\tt Enter file in which to save the key ...
    :}'', then press ``Enter'' twice if you do not wish to create a passphrase at this time or type your selected
    passphrase if you do).   Type {\tt man ssh-keygen} and talk with your partner to learn more about this program.
    What files does {\tt ssh-keygen} produce?  Where does this program store these files by default?

    Once you and your partner have created your ssh keys, you should, if necessary, raise your hand to invite the course
    instructor to help you with the next steps. First, you must log into Bitbucket and look in the right corner for an
    account avatar with a down arrow.  Click on this blue link and then select the ``Manage account'' option. Now,
    scroll down until you find the ``SSH keys'' option and upload your ssh key to Bitbucket. You can copy your SSH key
    by going to the terminal and typing ``{\tt cat \textasciitilde{}/.ssh/id\_rsa.pub}'' command.

  \item Now, you need to test to see if you can authenticate with the Bitbucket servers.  First, show the course
    instructor that you have correctly configured your Bitbucket account.  Now, ask your instructor to share the
    course's Git repository with you.  Open a terminal window on your workstation and change into the directory where
    you will store your files for this course.  If needed, you can open a terminal window and type the command ``{\tt
      mkdir cs441S2016}'', thus making a {\tt cs441S2016/} directory that will contain the Git repository that the
      instructor will always use to share files with you.  Once you have done so and, additionally, asked the course
      instructor to share this repository with you, please type the following command ``{\tt git clone
      git@bitbucket.org:gkapfham/cs441s2016-share.git}''.

    If everything worked correctly, you should be able to download all of the files that you will need for this
    practical assignment. Please resolve any problems that you encountered by first reviewing the Bitbucket
    documentation and then discussing the matter with a teaching assistant.  If you are still not able to run ``{\tt git
      clone}'', then please see the course instructor and work with your partner to resolve this issue. One problem that
      students commonly confront is the incorrect addition of SSH key to the Bitbucket system.

  \item Using your terminal window, you should browse the files that are in this Git repository.  In particular, please
    look in the {\tt cs441s2016-share/labs/lab1/} directory and use {\tt gvim} to study the Java classes that
    you find.  Remember, the ``{\tt cd}'' command allows you to change into a directory. Make sure that both you and
    your partner can access these files and then make sure to copy them to the shared repository, which you will
    setup in the following step.

  \end{enumerate}

  It is worth noting that many of the students in this class may have already configured {\tt git} and Bitbucket as a
  means for sharing source code. If this is the case for both you and your partner, then you may skip all of the previous
  steps and create your own repository that you will use to complete this laboratory assignment. Students who have not
  used Bitbucket in the past should work with the course instructor and their partner to ensure that they can use this
  important tool throughout the remainder of the semester. Again, once you have gained access to the {\tt
    cs441S2016-share} repository, then you and your partner should create a new repository called {\tt
  cs441S2016-lab01-<first user name>-<second use name>}. Finally, make sure that both individuals in your partnership, and
  the course instructor, have access to the repository that you will use for this assignment. Students who have questions
  about the use of the Git version control system should ensure that they have resolved them before the completion of this
  first laboratory assignment.

\section*{Exploring Client-Server Communication}

Please find the {\tt labs/lab1/} directory in our course's repository and work with your partner to study and understand
the source code for the files called {\tt FactorizationSocketClient} and {\tt FactorizationSocketServer}. What are the
key aspects of how these two Java classes support client-server communication? For instance, you will notice that these
Java classes specify items like a ``host'' and a ``port'' --- what are the meanings of these terms? Can you draw a
technical diagram that illustrates the general interaction between a client and a server, customized for these two
classes that uses sockets? Please see the instructor if you have questions about these issues.

You and your partner should work together to fully comment all of the source code in these two Java classes, making sure
that you correctly explain how all of the methods and lines of code work. If you consult online references to support
the development of your comments, then please include a reference to those sources in the comments. Once you have
finished compiling these two classes, please run the {\tt FactorizationSocketServer} in one terminal window and then
attempt to have the {\tt FactorizationSocketClient} connect to it. For now, you should run both of these programs on the
same computational node. Once you have gotten the client-server communication to work correctly, record the output of
the client and the server to demonstrate that they are working. If you are interested in exploring this further, please
try to run the client on the computer of one partner and the server on the computer of another. Students should be aware
that, depending on their setup, they may need to enhance these two Java classes to fully support remote communication.

Finally, you will notice that the {\tt FactorizationSocketClient} and {\tt FactorizationSocketServer} have some
``hard-coded'' values in them. For instance, the port on which socket communication takes place can only be defined if a
programmer changes the value in the source code and recompiles the Java class. Since this is not ideal from either a
usability or an experimentation perspective, you and your partner should revise the source code so that it accepts
command-line arguments for any values that you deem to be configurable by the user. While you may use the {\tt args[]}
array to accept these parameters, students are encouraged to consider the use of the JCommander library available at
\url{http://jcommander.org/}. Please see the instructor with questions about this task.

\section*{Implementing and Using a Benchmarking Framework}

Now, you and your partner should study the code for the {\tt LSSocketClient} and the {\tt LSSocketServer} --- these Java
classes will serve as the start of a simple benchmarking framework that you and your partner will implement. The ``{\tt
LS}'' abbreviated in the names of these classes is for the words ``Large'' and ``Small'' --- what about the
client-server interaction in this system can be classified in this fashion? After discussing this question with your
partner, you should implement six additional Java classes respectively called, {\tt SLSocketClient}, {\tt
SLSocketServer}, {\tt SSSocketClient}, {\tt SSSocketServer}, {\tt LLSocketClient}, and {\tt LLSocketServer}. As you are
implementing all of these Java classes make sure that the ways in which these clients and servers interact adhere to the
names that you have assigned to the classes. Overall, these programs will enable you to establish some ``baseline''
measurements for the cost of performing client-server communication on a single computer. Next, you and your partner
should implement one final program that is more like the {\tt FactorizationSocketClient} and {\tt
FactorizationSocketServer} --- that is, it should perform some type of ``useful'' computation on the server whenever the
client makes the appropriate request. Teams that are unsure of what type of additional computation to
implement should take time to brainstorm ideas with the instructor.

As you are creating your programs, make sure that you give them command-line arguments that will effectively support
running experiments and collecting data about the response time of the client-server interaction. Once you have
implemented all of the Java classes that you will use in your benchmarking framework, you and your partner should divide
up the work associated with running your experiments. For each of the four ``baseline'' client-server configurations,
you should run the client multiple times to collect many timings and then calculate, at minimum, arithmetic means and
standard deviations. Then, you should run both of the ``useful'' computations (i.e., the one for factorization and the
one that you have implemented with your partner), again collecting multiple timings and calculating arithmetic means and
standard deviations. For this laboratory assignment, you and your partner should focus on benchmarking local
client-server communication on the same computational node. However, teams that wish to potentially earn extra credit
should consider also running their experiments to measure the costs of remote communication.

You and your partner should organize all of your empirical results into tables of data. While not absolutely required,
you may consider preparing graphs of your results using the R language for statistical computation. Next, you should
analyze the results in attempt to find and explain patterns in the data. Overall, what do your results show you about
the cost of performing client-server communication? Now that you are finished running these experiments, can you
identify any ways in which you could have improved the performance of the client or the server? Finally, you and your
partner should write a detailed report, using either the Markdown or the \LaTeX~text formatting language, that
introduces the design of your experiment and your research questions, explains how you conducted the experiments, and
then presents and analyzes the results.

\section*{Summary of the Required Deliverables}

This assignment invites you to submit printed and signed versions of the following deliverables:

\begin{enumerate}

    \item Using screenshots and concrete examples, a description of the key features provided by zsh

    \item A commentary on the features that zsh provides that your previous shell did not

    \item A tutorial that explains how to download, install, and configure oh-my-zsh, including how to:

    \item A screenshot showing the final configuration of your zsh prompt in all relevant contexts

    \item A description of the challenges that you encountered when completing this assignment.

\end{enumerate}

In adherence to the Honor Code, students should complete this assignment on an individual basis. While it is appropriate
for students in this class to have high-level conversations about the assignment, it is necessary to distinguish
carefully between the student who discusses the principles underlying a problem with others and the student who produces
assignments that are identical to, or merely variations on, someone else's work.  With the exception of the provided
source code, deliverables that are otherwise nearly identical to the work of others will be taken as evidence of violating the
\mbox{Honor Code}. This means that, for instance, all of the other comments, source code, data, and report should be the
original of the two members of the partnership.

\end{document}
