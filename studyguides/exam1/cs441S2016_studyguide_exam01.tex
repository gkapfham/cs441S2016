% Typical usage (all UPPERCASE items are optional):
%       \input 111pre
%       \begin{document}
%       \MYTITLE{Title of document, e.g., Lab 1\\Due ...}
%       \MYHEADERS{short title}{other running head, e.g., due date}
%       \PURPOSE{Description of purpose}
%       \SUMMARY{Very short overview of assignment}
%       \DETAILS{Detailed description}
%         \SUBHEAD{if needed} ...
%         \SUBHEAD{if needed} ...
%          ...
%       \HANDIN{What to hand in and how}
%       \begin{checklist}
%       \item ...
%       \end{checklist}
% There is no need to include a "\documentstyle."
% However, there should be an "\end{document}."
%
%===========================================================
\documentclass[11pt,twoside,titlepage]{article}
%%NEED TO ADD epsf!!
\usepackage{threeparttop}
\usepackage{graphicx}
\usepackage{latexsym}
\usepackage{color}
\usepackage{listings}
\usepackage{fancyvrb}
%\usepackage{pgf,pgfarrows,pgfnodes,pgfautomata,pgfheaps,pgfshade}
\usepackage{tikz}
\usepackage[normalem]{ulem}
\tikzset{
    %Define standard arrow tip
%    >=stealth',
    %Define style for boxes
    oval/.style={
           rectangle,
           rounded corners,
           draw=black, very thick,
           text width=6.5em,
           minimum height=2em,
           text centered},
    % Define arrow style
    arr/.style={
           ->,
           thick,
           shorten <=2pt,
           shorten >=2pt,}
}
\usepackage[noend]{algorithmic}
\usepackage[noend]{algorithm}
\newcommand{\bfor}{{\bf for\ }}
\newcommand{\bthen}{{\bf then\ }}
\newcommand{\bwhile}{{\bf while\ }}
\newcommand{\btrue}{{\bf true\ }}
\newcommand{\bfalse}{{\bf false\ }}
\newcommand{\bto}{{\bf to\ }}
\newcommand{\bdo}{{\bf do\ }}
\newcommand{\bif}{{\bf if\ }}
\newcommand{\belse}{{\bf else\ }}
\newcommand{\band}{{\bf and\ }}
\newcommand{\breturn}{{\bf return\ }}
\newcommand{\mod}{{\rm mod}}
\renewcommand{\algorithmiccomment}[1]{$\rhd$ #1}
\newenvironment{checklist}{\par\noindent\hspace{-.25in}{\bf Checklist:}\renewcommand{\labelitemi}{$\Box$}%
\begin{itemize}}{\end{itemize}}
\pagestyle{threepartheadings}
\usepackage{url}
\usepackage{wrapfig}
% \usepackage{hyperref}
\usepackage[hidelinks]{hyperref}
%=========================
% One-inch margins everywhere
%=========================
\setlength{\topmargin}{0in}
\setlength{\textheight}{8.5in}
\setlength{\oddsidemargin}{0in}
\setlength{\evensidemargin}{0in}
\setlength{\textwidth}{6.5in}
%===============================
%===============================
% Macro for document title:
%===============================
\newcommand{\MYTITLE}[1]%
   {\begin{center}
     \begin{center}
     \bf
     CMPSC 280\\Principles of Software Development\\
     Fall 2015\\
     \medskip
     \end{center}
     \bf
     #1
     \end{center}
}
%================================
% Macro for headings:
%================================
\newcommand{\MYHEADERS}[2]%
   {\lhead{#1}
    \rhead{#2}
    \immediate\write16{}
    \immediate\write16{DATE OF HANDOUT?}
    \read16 to \dateofhandout
    \lfoot{\sc Handed out on \dateofhandout}
    \immediate\write16{}
    \immediate\write16{HANDOUT NUMBER?}
    \read16 to\handoutnum
    \rfoot{Handout \handoutnum}
   }

%================================
% Macro for bold italic:
%================================
\newcommand{\bit}[1]{{\textit{\textbf{#1}}}}

%=========================
% Non-zero paragraph skips.
%=========================
\setlength{\parskip}{1ex}

%=========================
% Create various environments:
%=========================
\newcommand{\PURPOSE}{\par\noindent\hspace{-.25in}{\bf Purpose:\ }}
\newcommand{\SUMMARY}{\par\noindent\hspace{-.25in}{\bf Summary:\ }}
\newcommand{\DETAILS}{\par\noindent\hspace{-.25in}{\bf Details:\ }}
\newcommand{\HANDIN}{\par\noindent\hspace{-.25in}{\bf Hand in:\ }}
\newcommand{\SUBHEAD}[1]{\bigskip\par\noindent\hspace{-.1in}{\sc #1}\\}
%\newenvironment{CHECKLIST}{\begin{itemize}}{\end{itemize}}

\begin{document}
\MYTITLE{Examination 1 Study Guide \\ Delivered: Tuesday, September 22, 2015 \\ Examination: Thursday, October 1, 2015, 11:00 am}

\section*{Introduction}

This course will have its first examination on Thursday, October 1, 2015 from 11:00 to 12:15 pm. The examination will be
``closed notes'' and ``closed book'' and it will cover the following content. Please review the ``Course Schedule'' on
the Web site for the course to see the content and slides that we have covered in the first module. You may post
questions about this material to Slack.

\begin{itemize}

  \itemsep 0in

  \item Chapters One through Three in SETP (i.e., introduction to the software engineering lifecycle)

  \item Chapters One through Three in MMM (i.e., challenges and solutions in software engineering)

  \item Your class notes, class activities, lecture slides, and the first four laboratory assignments

  \item Knowledge of the basic commands necessary for using {\tt git} and Bitbucket; basic understanding of the Markdown
    syntax and the use of associated command-line tools such as {\tt pandoc}

\end{itemize}

\vspace*{-.05in}

\noindent The examination will include a mix of questions that will require you to draw and/or comment on a diagram,
write a short answer, explain and/or write a source code segment, or give and comment on a list of concepts or points.
The emphasis will be on the following list of illustrative topics:

\vspace*{-.05in}
\begin{itemize}

  \itemsep 0in

  \item The state-of-the-art and the key challenges within the field of software engineering, with a focus on the steps
    of problem solving and the meaning of terms like ``defect'' and ``quality''.

  \item The phases of the software development lifecycle and the ways in which different software process models (e.g.,
    the spiral model or the V model) connect and interpret these phases.

  \item The key strengths and weaknesses of the different software development process models (e.g., one drawback of
    the waterfall model is its focus on documents and its lack of explicit iteration).

  \item Key terms such as ``verification'' and ``validation'' and ``incremental'' and ``iterative''.

  \item How to use activity graphs to track progress and plan a software development project. Additionally, an
    understanding of the ways in which managers will estimate the deadlines for completing a software system (e.g.,
    using data mining algorithms to predict project characteristics such as anticipated costs and the likelihood of an
    on-time completion).

  \item The roles that members of a software team may play and the ways in which individual personalities
    and characteristics may equip certain people to work on specific tasks.

  \item How different types of software engineering tasks exhibit different relationships between the time-to-completion
    and the number of workers assigned to finish the task.

  \item Lessons learned from working in a team to specify, design, implement, test, document, and release a programming
    systems product during our laboratory sessions.

\end{itemize}

\vspace*{-.05in}
\noindent Minimal partial credit may be awarded for the questions that require a student to write a short answer. You
are strongly encouraged to write short, precise, and correct responses to all of the questions. When you are taking the
examination, you should do so as a ``point maximizer'' who first responds to the questions that you are most likely to
answer correctly for full points. Please keep the time limitation in mind as you are absolutely required to submit the
examination at the end of the class period unless you have written permission for extra time from a member of the
Learning Commons. Students who do not submit their examination on time will have their overall point total reduced.
Please see the course instructor if you have questions about any of these policies.

\vspace*{-.1in}
\section*{Review the Honor Code}
\vspace*{-.06in}

\noindent Students are required to fully adhere to the Honor Code during the completion of this examination. More
details about the Allegheny College Honor Code are provided on the syllabus. Students are strongly encouraged to
carefully review the full statement of the Honor Code before taking \mbox{this test}.

\noindent The following provides you with a review of Honor Code statement from the course syllabus:

The Academic Honor Program that governs the entire academic program at Allegheny College is described in the Allegheny
Academic Bulletin.  The Honor Program applies to all work that is submitted for academic credit or to meet non-credit
requirements for graduation at Allegheny College.  This includes all work assigned for this class (e.g., examinations,
laboratory assignments, and the final project).  All students who have enrolled in the College will work under the Honor
Program.  Each student who has matriculated at the College has acknowledged the following pledge:

\vspace*{-.11in}
\begin{quote}
  I hereby recognize and pledge to fulfill my responsibilities, as defined in the Honor Code, and to maintain the
  integrity of both myself and the College community as a whole.
\end{quote}
\vspace*{-.11in}

\vspace*{-.15in}
\section*{Strategies for Studying}
\vspace*{-.05in}

As you study for this examination, you are encouraged to form study groups with individuals who were previously a
member of one of your software development teams during a laboratory session. You can collaborate with these individuals
to ensure that you understand all of the key concepts mentioned on this study guide. Additionally, students are
encouraged to create a Slack channel that can host questions and answers that arise as you are studying for the test.
Even though the course instructor will try to, whenever possible, answer review questions that students post in this
channel, you are strongly encouraged to answer the questions posted by your colleagues as this will also help you to
ensure that you fully understand the material.

When studying for the test, don't forget that the Web site for our course contains mobile-ready slides that will provide
you with an overview of the key concepts that we discussed in the first module. You can use the color scheme in the
slides to notice points where we, for instance, completed an in-class activity, discussed a key point, or made reference
to additional details available in the SETP and MMM textbooks. Finally, students are strongly encouraged to schedule a
meeting during the course instructor's office hours so that we can resolve any of your questions about the material and
ensure that you have the knowledge and skills necessary for doing well on this examination. Remember, while the test is
taken individually, your review for it can be done collaboratively!

\end{document}
