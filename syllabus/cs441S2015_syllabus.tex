%!TEX root=cs441S2016_syllabus.tex

%!TEX root=cs441S2016_syllabus.tex

% Typical usage (all UPPERCASE items are optional):
%       \input 580pre
%       \begin{document}
%       \MYTITLE{Title of document, e.g., Lab 1\\Due ...}
%       \MYHEADERS{short title}{other running head, e.g., due date}
%       \PURPOSE{Description of purpose}
%       \SUMMARY{Very short overview of assignment}
%       \DETAILS{Detailed description}
%         \SUBHEAD{if needed} ...
%         \SUBHEAD{if needed} ...
%          ...
%       \HANDIN{What to hand in and how}
%       \begin{checklist}
%       \item ...
%       \end{checklist}
% There is no need to include a "\documentstyle."
% However, there should be an "\end{document}."
%
%===========================================================
\documentclass[11pt,twoside,titlepage]{article}
%%NEED TO ADD epsf!!
\usepackage{threeparttop}
\usepackage{graphicx}
\usepackage{latexsym}
\usepackage{color}
\usepackage{listings}
\usepackage{fancyvrb}
%\usepackage{pgf,pgfarrows,pgfnodes,pgfautomata,pgfheaps,pgfshade}
\usepackage{tikz}
\usepackage[normalem]{ulem}
\tikzset{
    %Define standard arrow tip
%    >=stealth',
    %Define style for boxes
    oval/.style={
           rectangle,
           rounded corners,
           draw=black, very thick,
           text width=6.5em,
           minimum height=2em,
           text centered},
    % Define arrow style
    arr/.style={
           ->,
           thick,
           shorten <=2pt,
           shorten >=2pt,}
}
\usepackage[noend]{algorithmic}
\usepackage[noend]{algorithm}
\newcommand{\bfor}{{\bf for\ }}
\newcommand{\bthen}{{\bf then\ }}
\newcommand{\bwhile}{{\bf while\ }}
\newcommand{\btrue}{{\bf true\ }}
\newcommand{\bfalse}{{\bf false\ }}
\newcommand{\bto}{{\bf to\ }}
\newcommand{\bdo}{{\bf do\ }}
\newcommand{\bif}{{\bf if\ }}
\newcommand{\belse}{{\bf else\ }}
\newcommand{\band}{{\bf and\ }}
\newcommand{\breturn}{{\bf return\ }}
\newcommand{\mod}{{\rm mod}}
\renewcommand{\algorithmiccomment}[1]{$\rhd$ #1}
\newenvironment{checklist}{\par\noindent\hspace{-.25in}{\bf Checklist:}\renewcommand{\labelitemi}{$\Box$}%
\begin{itemize}}{\end{itemize}}
\pagestyle{threepartheadings}
\usepackage{url}
\usepackage{wrapfig}
% removing the standard hyperref to avoid the horrible boxes
%\usepackage{hyperref}
\usepackage[hidelinks]{hyperref}
% added in the dtklogos for the bibtex formatting
\usepackage{dtklogos}
%=========================
% One-inch margins everywhere
%=========================
\setlength{\topmargin}{0in}
\setlength{\textheight}{8.5in}
\setlength{\oddsidemargin}{0in}
\setlength{\evensidemargin}{0in}
\setlength{\textwidth}{6.5in}
%===============================
%===============================
% Macro for document title:
%===============================
\newcommand{\MYTITLE}[1]%
   {\begin{center}
     \begin{center}
     \bf
     CMPSC 441\\Distributed Systems\\
     Spring 2016\\
     \medskip
     \end{center}
     \bf
     #1
     \end{center}
}
%================================
% Macro for headings:
%================================
\newcommand{\MYHEADERS}[2]%
   {\lhead{#1}
    \rhead{#2}
    %\immediate\write16{}
    %\immediate\write16{DATE OF HANDOUT?}
    %\read16 to \dateofhandout
    \def \dateofhandout {January 19, 2016}
    \lfoot{\sc Handed out on \dateofhandout}
    %\immediate\write16{}
    %\immediate\write16{HANDOUT NUMBER?}
    %\read16 to\handoutnum
    \def \handoutnum {1}
    \rfoot{Handout \handoutnum}
   }

%================================
% Macro for bold italic:
%================================
\newcommand{\bit}[1]{{\textit{\textbf{#1}}}}

%=========================
% Non-zero paragraph skips.
%=========================
\setlength{\parskip}{1ex}

%=========================
% Create various environments:
%=========================
\newcommand{\PURPOSE}{\par\noindent\hspace{-.25in}{\bf Purpose:\ }}
\newcommand{\SUMMARY}{\par\noindent\hspace{-.25in}{\bf Summary:\ }}
\newcommand{\DETAILS}{\par\noindent\hspace{-.25in}{\bf Details:\ }}
\newcommand{\HANDIN}{\par\noindent\hspace{-.25in}{\bf Hand in:\ }}
\newcommand{\SUBHEAD}[1]{\bigskip\par\noindent\hspace{-.1in}{\sc #1}\\}
%\newenvironment{CHECKLIST}{\begin{itemize}}{\end{itemize}}


\usepackage[compact]{titlesec}

\begin{document}
\MYTITLE{Syllabus}
\MYHEADERS{Syllabus}{}

\subsection*{Course Instructor}
Dr.\ Gregory M.\ Kapfhammer\\
\noindent Office Location: Alden Hall 108 \\
\noindent Office Phone: +1 814-332-2880 \\
\noindent Email: \url{gkapfham@allegheny.edu} \\
\noindent Twitter: \url{@GregKapfhammer} \\
\noindent Web Site: \url{http://www.cs.allegheny.edu/sites/gkapfham/}

\subsection*{Instructor's Office Hours}

\begin{itemize}
  \itemsep 0em

  \item Tuesday: 11:00 am -- 12:00 noon (15 minute time slots) {\em and} \\ \hspace*{.6in}
    2:30 pm - 3:30 pm (15 minute time slots)

  \item Wednesday: 4:30 pm -- 5:30 pm (10 minute time slots)

  \item Thursday: 11:00 am -- 12:00 noon (15 minute time slots) {\em and} \\ \hspace*{.68in}
    1:00 pm - 3:00 pm (15 minute time slots)

  \item Friday: 3:00 pm -- 5:00 pm (15 minute time slots)

\end{itemize}

\noindent To schedule a meeting with me during my office hours, please visit my Web site and click the ``Schedule'' link
in the top right-hand corner. Now, you can browse my office hours or schedule an appointment by clicking the correct
link and then reserving an open time slot. Students are also encouraged to post appropriate questions to a channel in
Slack, which is available at \url{https://CMPSC441Spring2016.slack.com}, and monitored by the course instructor.

\subsection*{Course Meeting Schedule}

Lecture, Discussion, and Group Work: Tuesday, Thursday, 11:00 am -- 12:15 pm\\
Laboratory Session: Monday, 2:30 pm -- 4:20 pm \\
Final Examination: Thursday, May 5, 2016 at 7:00 pm

\subsection*{Course Description}

\begin{quote}

An examination of the principles and paradigms associated with the design, implementation, and analysis of distributed
systems. Topics include the characterization of distributed system models, remote communication, distributed scheduling,
synchronization and mutual exclusion, naming and time, consistency and replication, and fault tolerance. Selected
distributed system development environments are discussed in the context of the above topics. One laboratory per week.
Prerequisites: CMPSC 280 or CMPSC 440 or permission of the instructor.

\end{quote}

\subsection*{Course Objectives}

The design, implementation, and use of distributed systems involves the application of many interesting theories,
techniques, methodologies, and tools.  This course has the objective to:

\begin{enumerate}

  \itemsep 0in

  \item Provide an overview of the nature and functions of distributed systems.

  \item Study the relationship between computer hardware and distributed systems.

  \item Understand the connections between operating systems and distributed systems.

  \item Enhance the understanding of the services that distributed systems provide to users.

  \item Explore key distributed system concepts (e.g., processes and remote communication).

  \item Study the algorithms used in distributed systems (e.g., load balancing and scheduling).

  \item Examine, in detail, the design of several important components of a distributed system.

  \item Develop a ``big picture'' understanding of the overall design of a distributed system.

  \item Develop a basic understanding of security and reliability issues in distributed systems.

  \item Enhance knowledge of the software tools used to design and implement distributed systems.

\end{enumerate}

\vspace*{-.075in}

\noindent Throughout the semester students also will enhance their ability to write and present ideas about distributed
systems in a clear and compelling fashion.  Students will gain practical experience in the design, implementation, and
analysis of distributed systems during laboratory sessions and a final project. Finally, students will develop a richer
understanding of the fascinating connections between distributed systems and other disciplines in the social and natural
sciences and the humanities.

\subsection*{Required Textbook}

\noindent{\em Distributed Systems: Principles and Paradigms}. Andrew S.\ Tanenbaum and Maarten van Steen.
Second Edition, ISBN: 0-13-239227-5, 686 pages, 2007. \\
(References to the textbook are abbreviated as ``DSPP'' on the course Web site).

\noindent
Students who want to improve their technical writing skills may consult the following books.

\noindent{\em BUGS in Writing: A Guide to Debugging Your Prose}. Lyn Dupr\'e. Second Edition,  ISBN-10: 020137921X,
ISBN-13: 978-0201379211, 704 pages, 1998.

\noindent{\em Writing for Computer Science}.  Justin Zobel. Second Edition,  ISBN-10: 1852338024, ISBN-13:
978-1852338022, 270 pages, 2004.

\noindent
Along with reading the required textbook, you will be asked to study additional articles from a wide variety of
conference proceedings, journals, and the popular press.

\subsection*{Class Policies}

\subsubsection*{Grading}

The grade that a student receives in this class will be based on the following categories. All percentages are
approximate and, if the need to do so presents itself, it is possible for the assigned percentages to change during the
academic semester.

\begin{center}
  \begin{tabular}{ll}
    Class Participation and Instructor Meetings & 10\% \\
    First Examination & 15\% \\
    Second Examination & 15\% \\
    Final Examination & 20\% \\
    Laboratory Assignments & 30\% \\
    Final Project & 10\%
  \end{tabular}
\end{center}

\noindent
These grading categories have the following definitions:
\vspace*{-.1in}

\begin{itemize}

    \item {\em Class Participation and Instructor Meetings}: All students are required to actively participate
      during all of the class sessions. Your participation will take forms such as answering questions about the
      required reading assignments, asking constructive questions of your group members, giving presentations, and
      leading a discussion session. Furthermore, all students are required to meet with the course instructor during
      office hours for a total of thirty minutes throughout the Spring 2016 semester.  These meetings must be scheduled
      through the course instructor's reservation system and documented on a meeting record that you submit on the
      day of the final examination. A student will receive an interim and final grade for this category.

    \item {\em First and Second Examinations}: The first and second interim examinations will cover all of the material
      in their associated modules, as outlined on the review sheet.  While the second examination is not cumulative, it
      will assume that a student has a basic understanding of the material that was the focus of the first examination.
      The date for the first and second examinations will be announced at least one week in advance of the scheduled
      date.  Unless prior arrangements are made with the course instructor, all students will be expected to take these
      examinations on the scheduled date and to complete them in the stated period of time.

    \item {\em Final Examination}: The final examination is a three-hour cumulative test.  By enrolling in this
      course, students agree that, unless there are extenuating circumstances, they will take the final examination
      at the date and time stated on the first page of the syllabus.

    \item {\em Laboratory Assignments}: These assignments invite students to explore different techniques for designing,
      implementing, evaluating, and documenting software solutions to challenging problems in the field of distributed
      systems.  Many of the assignments will require students to write programs, conduct experiments, and collect,
      analyze, and write about data sets.  To best ensure that students are ready to develop software in both other
      classes at Allegheny College and after graduation, students will complete assignments either on an individual
      basis or in teams.  When teamwork is required, the instructor will often assign individuals to teams.

    \item {\em Final Project}: This project will present you with the description of an distributed systems problem and
      ask you to design and implement a correct and carefully evaluated solution. Completion of the final project will
      require you to apply all of the knowledge and skills that you have accumulated during the course of the semester
      to solve a problem and, whenever possible, make your solution and results publicly available in a free and open
      fashion.

\end{itemize}

\subsubsection*{Assignment Submission}

All assignments will have a stated due date. Electronic versions of the laboratory and the final project assignments
must be submitted to the version control repository that the student creates at the start of the semester.
Additionally, the printed version of the assignment is to be turned in at the beginning of the class on that due date;
the printed materials must be dated and signed with the Honor Code pledge of the student(s) completing the work.  Late
assignments will be accepted for up to one week past the assigned due date with a 15\% penalty. All of the late
assignments must be turned in at the beginning of the session that is scheduled one week after the due date. Unless
special arrangements are made with the instructor, no work will be accepted after the late deadline. For any assignment
completed in a group, students must also turn in a one-page document that describes, in detail, each group member's
contribution to the submitted deliverables.

\subsubsection*{Course Attendance}

It is mandatory for all students to attend all of the class and laboratory sessions. If, due to extenuating
circumstances, you will not be able to attend a session, then, whenever possible, please see the course instructor at
least one week in advance to describe your situation.  Students who miss more than five unexcused sessions will have
their final grade in the course reduced by one letter grade. Students who miss more than ten of the aforementioned
events will fail the course.

\subsubsection*{Use of Laboratory Facilities}

Throughout the semester, we will experiment with many different software tools that computer scientists use during the
design, implementation, evaluation, and interaction with operating systems.  The course instructor and the
department's systems administrator have invested a considerable amount of time to ensure that our laboratories support
the completion of both the laboratory assignments and the final project.  To this end, students are required to complete
all assignments and the final project while using the department's laboratory facilities. The course instructor and the
systems administrator normally do not assist students in configuring their personal computers.

\subsubsection*{Class Preparation}

In order to minimize confusion and maximize learning, students must invest time to prepare for the class discussions,
lectures, and laboratory sessions.  During the class periods, the course instructor will often pose demanding questions
that could require group discussion, the creation of a program or data set, a vote on a thought-provoking issue, or a
group presentation.  Only students who have prepared for class by reading the assigned material and reviewing the
current laboratory and practical assignments will be able to effectively participate in these discussions.

More importantly, only prepared students will be able to acquire the knowledge and skills that are needed to be
successful in this course, subsequent courses, and the field of computer science.  In order to help students remain
organized and effectively prepare for classes, the course instructor will maintain a class schedule with reading
assignments and presentation slides.   During the class sessions students will also be required to download, use, and
modify programs and data sets that are made available through means such as the course Web site and a version control
repository.

\subsubsection*{Seeking Assistance}

Students who are struggling to understand the knowledge and skills developed in a class or laboratory session session
are encouraged to seek assistance from the course instructor. Throughout the semester, students should, within the
bounds of the Honor Code, ask and answer questions on the Slack site for our course; please request assistance from the
instructor first through Slack before sending an email. Students who need the instructor's assistance must schedule a
meeting through his Web site and come to the meeting with all of the details needed to discuss their question.

\subsubsection*{Using Email}

Although we will primarily use Slack for class communication, I will sometimes use email to send announcements about
important matters such as changes in the schedule. It is your responsibility to check your email at least once a day and to
ensure that you can reliably send and receive emails. This class policy is based on the statement about the use of email that
appears in {\em The Compass}, the College's student handbook; please see the instructor if you do not have this
handbook.

\subsubsection*{Disability Services}

The Americans with Disabilities Act (ADA) is a federal anti-discrimination statute that provides comprehensive civil
rights protection for persons with disabilities.  Among other things, this legislation requires all students with
disabilities be guaranteed a learning environment that provides for reasonable accommodation of their disabilities.
Students with disabilities who believe they may need accommodations in this class are encouraged to contact Disability
Services at 332-2898.  Disability Services is part of the Learning Commons and is located in Pelletier Library.
Please do this as soon as possible to ensure that approved accommodations are implemented in a timely fashion.

\subsubsection*{Honor Code}

The Academic Honor Program that governs the entire academic program at Allegheny College is described in the Allegheny
Course Catalogue.  The Honor Program applies to all work that is submitted for academic credit or to meet non-credit
requirements for graduation at Allegheny College.  This includes all work assigned for this class (e.g., examinations,
laboratory assignments, and the final project).  All students who have enrolled in the College will work under the Honor
Program.  Each student who has matriculated at the College has acknowledged the following pledge:

\vspace*{-.1in}
\begin{quote}
I hereby recognize and pledge to fulfill my responsibilities, as defined in the Honor Code, and to maintain the
integrity of both myself and the College community as a whole.
\end{quote}
\vspace*{-.15in}

\noindent It is recognized that an important part of the learning process in any course, and particularly one in
computer science, derives from thoughtful discussions with teachers and fellow students.  Such dialogue is encouraged.
However, it is necessary to distinguish carefully between the student who discusses the principles underlying a problem
with others and the student who produces assignments that are identical to, or merely variations on, someone else's
work.  While it is acceptable for students in this class to discuss their programs, data sets, and reports with their
classmates, deliverables that are nearly identical to the work of others will be taken as evidence of violating the
\mbox{Honor Code}.

\subsection*{Welcome to an Adventure in Distributed Systems}

In reference to software, Frederick P.\ Brooks, Jr.\ wrote in chapter one of {\em The Mythical Man Month}, ``The magic
of myth and legend has come true in our time.'' Software is a pervasive aspect of our society and distributed systems
are one of the crucial components that enables people to interact with a wide range of hardware and software. At the
start of this class, I invite you to participate in this adventure in the design, implementation, and evaluation of
distributed systems!

\end{document}
